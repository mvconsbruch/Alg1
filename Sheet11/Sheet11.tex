\documentclass[a4paper,11pt]{article}
\pagenumbering{arabic}
\usepackage{../environment}
\usepackage{blindtext}
\usepackage{hyperref}
\usepackage{../quiver}

\begin{document}

\begin{center}
    \huge{Solutions to Sheet 11}
\end{center}

\exercise{1}
Let $k$ be a field and let $A,B$ be two finitely generated $k$-algebras.
Show 
\begin{equation*}
    \dim(A \otimes_k B) = \dim(A) + \dim(B).
\end{equation*}

\textbf{Solution.} Remember noether normalization? It tells us that given
any finitely generated $k$-algebra $A$ of dimension $n$, there is some 
integral extension $k[x_1, \dots, x_n] \inj A$. Similarly, $B$ arises
as an integral extension $k[y_1, \dots, y_m] \inj B$. 
We now have an injection
\begin{equation*}
    k[x_1, \dots, x_n, y_1, \dots, y_m] = k[x_1, \dots, x_n] \otimes_k k[y_1, \dots, y_m] 
    \inj A \otimes_k k[y_1, \dots, y_m] \inj A \otimes_k B.
\end{equation*}
Note that both maps are base changes of integral maps, thereby integral
themself. To see this, look at the following diagram where every
square is co-cartesian (i.e., in every square, the top-right is isomorphic to
the tensor product along the corners)
\begin{equation*}
\begin{tikzcd}[ampersand replacement=\&]
	A \& {A \otimes_k k[y_1, \dots, y_m] } \& {A \otimes_k B} \\
    {k[x_1, \dots, x_n]} \& {k[x_1, \dots, x_n, y_1, \dots, y_m]} \& {k[x_1, \dots, x_n] \otimes_k B} \\
	k \& {k[y_1, \dots, y_m]} \& B
	\arrow[from=3-2, to=2-2]
	\arrow[from=3-1, to=3-2]
	\arrow[from=3-1, to=2-1]
	\arrow[from=2-1, to=2-2]
	\arrow["{\text{integral}}"', hook, from=3-2, to=3-3]
	\arrow[from=3-3, to=2-3]
	\arrow["{\therefore \text{integral}}"', from=2-2, to=2-3]
	\arrow["{\text{integral}}"',hook, from=2-1, to=1-1]
	\arrow["{\therefore \text{integral}}"', from=2-2, to=1-2]
	\arrow[from=1-1, to=1-2]
	\arrow["{\therefore \text{integral}}"', from=1-2, to=1-3]
	\arrow["{\therefore \text{integral}}"', from=2-3, to=1-3]
\end{tikzcd}
\end{equation*}

Hence the map above is integral. As integral homomorphisms 
preserve dimension, we find $\dim(A \otimes_k B) = n+m = \dim(A) + \dim(B)$. 
Indeed, by going up we find that $\dim(A \otimes_k B) \geq n+m$. If 
the inequality was strict, we could apply Noether normalization again, 
eventually finding an integral extension of the form
$k[x_1, \dots, x_{n+m}] \inj k[x_1, \dots, x_{n+m+1}]$, which is 
absurd.

\exercise{2}
Let $k$ be a field, and consider the $k$-algebra morphism
\begin{equation*}
    \phi: k[x,y]/(y^2 - x^3) \to k[t], \quad x \mapsto t^2, y \mapsto t^3.
\end{equation*}
Show that $\phi$ is finite, induces a bijection on $\spec$ and is not
an isomorphism.

\textbf{Solution.} This is not an isomorphism because $t$ does not lie 
in the image. 

To show that $\phi$ induces a bijection on spectra, note that it is 
an isomorphism if we invert $x$ and $t$:
\begin{equation*}
    k[x^{\pm 1},y]/(y^2 - x^3) = k[x^{\pm 1}, x^{3/2}] 
    = k[x^{\pm \frac 12}] \cong k[t^{\pm 1}] ,\quad x^{\frac 12} \mapsto t.
\end{equation*}
In other words, restricting $\spec(\phi)$ to $\spec(A)\setminus \{(x)\}$
yields an isomorphism to $\spec(k[t])\setminus \{(t)\}$. But one easily
checks that the preimage of $(t)$ is given by the ideal generated by 
$(x)$, hence we have a bijection on spectra. (Geometrically, $\phi$
gives a parametrization of the cusp, given by $t = x/y$. In particular
$t = 0$ implies $x = 0$. This is one standard example of normalization)

To show finiteness, note that $(1, t, t^2, \dots)$ generates 
$k[t]$ as an $k[x,y]/(y^2-x^3)$-module. But $t^2 = x \cdot 1 \in 
k[x,y]/(y^2-x^3) \cdot 1$, so $(1,t)$ is a generating tuple. Hence the 
map is finite.

\exercise{3}
In this exercise we denote by $\MinSpec(A)$ the set of minimal prime ideals
of a ring $A$. 
\begin{enumerate}
    \item Let $A_1, \dots, A_n$ be rings and let $B$ be their product.
        Show that 
        \begin{equation*}
            \MinSpec(B) = \bigcup_{i = 1}^n \MinSpec(A_i).
        \end{equation*}
    \item Let $f: A\to B$ be an injective and integral ring homomorphism.
        Show that the inclusion
        \begin{equation*}
            \MinSpec(A) \subseteq \spec(f)(\MinSpec(B))
        \end{equation*}
        and give an example where the inclusion is strict. 
\end{enumerate}
\textbf{Solution.} A module $M$ over a product of rings 
$A_1, \dots, A_n$ is the same as modules $M_i$ over each of the rings 
$A_i$. Indeed, set $M_i = e_i M$ with $e_i \in A_1 \times \dots \times A_n$
the $i$-th standard entry. Now $e_j$ annihilates $e_i$ for $i \neq j$
and one can check that $M \cong e_1 M \times \dots \times e_n M$. 

For part $1$, this yields that there is an inclusion preserving bijection
$\spec(B) = \bigcup_{i=1}^n \spec(A_i)$. Indeed, any ideal is of the 
form $I = I_1 \times \dots \times I_n$, and for this to be prime we 
need $I_i = \fp \in \spec(A_i)$ for some $1 \leq i \leq n$ and $I_j = A_j$ for
all $j \neq i$. One easily checks that  all those ideals are prime. And if,
say, $I_1 \subsetneq A_1$ and $I_2 \subsetneq A_2$ are proper ideals, then
$(1,0) \not \in I_1 \times I_2$ and $(0,1) \not \in I_1 \times I_2$, but $(1,0)
\cdot (0,1) = (0,0) \in I_1 \times I_2$, so $I_1 \times I_2$ has no chance to
be prime.

For part 2, by lying over we have that $\spec(f)$ is surjective. So given 
any prime $\fp \in \spec(A)$ we find some $\fq \in \spec(B)$ with
$f^{-1}(\fq) = \fp$. But now there is some minimal prime $\fq' \subseteq \fq$,
and we find $f^{-1}(\fq') \subseteq f^{-1}(\fq) = \fp$. But by minimality of 
$\fp$ this implies $f^{-1}(\fq') = \fp$. Hence every minimal prime
of $A$ arises as the preimage of some minimal prime of $B$. This is what we 
had to show. 

\exercise{4}
Let $k$ be an algebraically closed field and let $Z \subset k^4$ be
the vanishing locus of the ideal
$(xz,yz,xw,yw) \subset k[x,y,z,w]$. Determine the irreducible components
of $Z$ and their intersections. 

\textbf{Solution.} Consider the projections $k[x,y,z,w] \to k[x,y]$ 
and $k[x,y,z,w] \to k[z,w]$. These yield a homomorphism $k[x,y,z,w] \to
k[x,y] \times k[z,w]$. The kernel is given by the intersection of the kerenels
of the two individual maps, which is $(z,w) \cap (x,y) = (xz,yz, xw, yw)$. 
This yields an injective homomorphism
$$A \coloneqq k[x,y,z,w]/(xz,yz,xw,yw) \to k[x,y] \times k[z,w].$$
One easily sees that this is finite. Indeed, the right hand side is generated
by $(1,0)$ and $(0,1)$ as $A$-modules. 
We are now in a position to apply the results of exercise 3. The set of minimal
primes of $k[x,y]$ is the singleton $\{(0)\}$. By 3.1 we find
\begin{equation*}
    \MinSpec(k[x,y] \times k[z,w] = \{(1,0), (0,1)\}.
\end{equation*}
We have $f^{-1}((1,0)) = (x,y)$ and $f^{-1}((0,1)) = (z,w)$. Hence 3.2 gives
\begin{equation*}
    \MinSpec(k[x,y,z,w]/(xz,yz,xw,yw)) \subseteq \{(x,y), (z,w)\}.
\end{equation*}
But there is at least one minimal prime and symmetry forces equality.

Now, as irreducible components are in bijection with minimal primes,
$\spec(A)$ has two irreducible components, given by
$V(x,y)$ and $V(z,w)$. Their intersection is given by 
$V(x,y,z,w) = \{(0,0,0,0)\}$, the set containing only the origin.



\contactend
\end{document}
