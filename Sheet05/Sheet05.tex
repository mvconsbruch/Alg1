\documentclass[a4paper,11pt]{article}
\pagenumbering{arabic}
\usepackage{../environment}

\begin{document}

\begin{center}
    \huge{Solutions to Sheet 5}
\end{center}

\exercise{1}
Let $A$ be a ring and let $\fa_1, \dots, \fa_n \subset A$ be ideals such
that $\bigcap_{i=1}^n \fa_i = \{0\}$. Assume that each ring 
$A / \fa_i$ is noetherian. Show that $A$ is noetherian. 

\textbf{Solution.}
When does this situation arise? One example are rings of the form
$A / \bigcap_i \fa_i$. 


\exercise{2}
Consider the matrix
\begin{equation*}
    S \coloneqq \begin{pmatrix}
        -36 & 14 & -24 \\
        18 & 6 & 12
    \end{pmatrix}.
\end{equation*}
Determine its elementary divisors and the kernel/cokernel of the map
$\Z^3 \xto S \Z^2$ (up to isomorphy).

\exercise{3}
Let $A$ be a ring, let $\fa \subset A$ be an ideal and let $M, N_i$, $i \in I$, be $A$-modules for some set $I$. 
\begin{enumerate}
    \item Show that there exists a unique isomorphism
        \begin{equation*}
            \Phi: \bigoplus_{i \in I} (N_i \otimes_A M) \to \left( \bigoplus_{i
                \in I} N_i \right ) \otimes_A M
        \end{equation*}
        such that $\Phi((\dots,0, n_i \otimes m, 0 \dots)) = (\dots, 0, n_i, 0,\dots) \otimes m$ for all $n_i \in N_i$, $i \in I$, $m \in M$. 
    \item Show that there exists a unique isomorphism
        \begin{equation*}
            \Psi: A / \fa \otimes_A M \to M/\fa M
        \end{equation*}
        such that $\Psi((a + \fa) \otimes m) \mapsto am + \fa M$ for all 
        $a \in A$, $m \in M$. 
\end{enumerate}

\textbf{Solution.} 
\begin{enumerate}
    \item By the unique property of the direct sum, defining $\Phi$ is the same
        as defining morphisms $\Phi_i: N_i \otimes_A M \to \left( \bigoplus_{i \in I}
            N_i \right) \otimes_A M$. Note that $N_i \otimes_A M$ is generated
            by elements of the form $n_i \otimes m$ (with $n_i \in N_i$ and $m
            \in M$), and the exercise already specifies how $\Phi_i$ is defined
            on those elements, namely by
            \begin{equation*}
                \Phi_i( n_i \otimes m) = (\dots, 0, n_i, 0, \dots) \otimes m \in
                \left( \bigoplus_{i \in I} N_i \right ) \otimes_A M.
            \end{equation*}
            One could check now that this is well defined (remember that the
            elements $n_i \otimes m$ are only defined up to the relation
            $(an) \otimes m \sim n \otimes (am)$). 
            But we use the universal property. The map is exactly the map that
            comes from the bilinear map
            $N_i \times M \to \left(\bigoplus_{i \in I} N_i \right) \otimes_A M$,
            $(n_i, m) \mapsto ( \dots,0, n_i,0, \dots ) \otimes m$.
            By construction, $\Psi_i$ is a bijection on its image (really, we 
            just put $0$s everywhere else), and the images of $\Psi_i$ have
            intersection $\{0\}$ and generate all of $\left( \bigoplus_{i \in I}
                N_i \right) \otimes M$. 
            Hence $\Psi$ is an isomorphism.

        \item Again, we use the universal property. The mapping
            \begin{equation*}
                A/\fa \times M \to M / \fa M, \quad (a + \fa, m) \mapsto 
                am + \fa M.
            \end{equation*}
            is well-defined and bilinear, which is easy to check. This gives the
            desired map $\Psi: A/\fa \otimes_A M \to M/\fa M$. It is surjective 
            as $\Psi(1 \otimes m) = m + \fa M$, and injective because if 
            $\Psi((a + \fa) \otimes m) = 0 + \fa M$, we have $am \in \fa M$.
            Hence $am = a' m'$ for some $a' \in \fa, m' \in M$. In particular,
            $$a \otimes m = 1 \otimes (am) = 1 \otimes (a' m') = a' \otimes m' = 0
            \in A/\fa \otimes_A M.$$ 
            This shows injectivity of $\Psi$, and we are done.
\end{enumerate}

\exercise{4}
Let $A$ be a ring and let $M,N$ be $A$-modules. A bilinear map 
$(-,-): M \times M \to N$ is called symmetric if $(m_1, m_2) = (m_2, m_1)$ for all
$m_1, m_2 \in M$. It is called alternating if $(m,m) = 0$ for all $m \in M$. 
\begin{enumerate}
    \item Show that there exists an $A$-module $\Sym^2_A(M)$ and a symmetric bilinear
        map $\iota: M \times M \to \Sym^2_A(M)$ with the following universal
        property: For every $A$-module $N$ and for every symmetric bilinear map
        $(-,-): M \times M \to N$ there exists a unique $A$-linear map $\Phi:
        \Sym^2_A(M) \to N$ usch that for all $m_1, m_2 \in M$
        \begin{equation*}
            (m_1, m_2) = \Psi(\iota(m_1, m_2)).
        \end{equation*}
        Construct similarly an $A$-module $\Lambda^2_A(M)$ with a universal
        alternating bilinear map $\gamma: M \times M \to \Lambda^2_A(M)$. 

    \item Show that $\Sym^2_A(A^n)$ and $\Lambda^2_A(A^n)$ are free $A$-modules
        of ranks $\frac{n(n+1)}2$ and $\frac{n(n-1)}2$. 
\end{enumerate}

\textbf{Solution.}
\begin{enumerate}
    \item
        Okay, the $\Sym$-construction should be somehow similar to the construciton
        of $\otimes$, and ideally all proofs of properties simply follow from
        the universal
        property of the tensor product. In the construction of the tensor product,
        $(m_1, m_2)$ corresponds to the image of $\phi(m_1 \otimes m_2)$ for 
        some suitable morphism $\phi$. Imposing that $(m_1, m_2) = (m_2, m_1)$ 
        corresponds to the statement that in $\Sym^2_A$, any morphism 
        should send $(m_1 \otimes m_2 - m_2 \otimes m_1)$ to zero. 
        Building on this, we define $\Sym^2_A(M)$ as $(M \otimes_A M)/ G$, where
        $G$ is the $A$-module generated by elements of the form $(m_1 \otimes m_2 -
        m_2 \otimes m_1)$. We check that this works. With the notation of the 
        exercise, we first obtain a morphism $\psi: M \otimes_A M \to N$ by
        the UP of the tensor product.
        \[\begin{tikzcd}[ampersand replacement=\&]
        	{M \times M} \&\& {M \otimes_A M} \\
        	\& N \&\& {\Sym_A^2(M)}
        	\arrow["{(-,-)}"', from=1-1, to=2-2]
        	\arrow["{(m_1, m_2) \mapsto m_1 \otimes m_2}", from=1-1, to=1-3]
        	\arrow["\psi", dashed, from=1-3, to=2-2]
        	\arrow["{\text{proj}}", from=1-3, to=2-4]
        	\arrow["\Psi", dashed, from=2-4, to=2-2]
        \end{tikzcd}\]
        By construction, we have $G \subset \ker \psi$, so by the universal property
        of kernels, $\psi$ extends uniquely to a morphism $\Psi: \Sym^2_A(M)
        \cong (M \otimes_A M) / G \to N$. 

        We define $\Lambda^2_A(M)$ similarly, this time we define $G$ as
        submodule of 
        $M \otimes_A M$ generated by elements of the form $(m \otimes m)$. 

    \item Note that similarly to vector spaces, $A^n \otimes_A A^n$ is the 
        free module generated over the basis $e_i \otimes e_j$. In the case of 
        $\Sym^2_A$, we need to quotient out the submodule generated by 
        elements of the form $m_1 \otimes m_2 - m_2 \otimes m_1$. 
\end{enumerate}


\contactend
\end{document}
