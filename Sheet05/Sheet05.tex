\documentclass[a4paper,11pt]{article}
\pagenumbering{arabic}
\usepackage{../environment}

\begin{document}

\begin{center}
    \huge{Solutions to Sheet 5}
\end{center}

\exercise{1}
Let $A$ be a ring and let $\fa_1, \dots, \fa_n \subset A$ be ideals such
that $\bigcap_{i=1}^n \fa_i = \{0\}$. Assume that each ring 
$A / \fa_i$ is noetherian. Show that $A$ is noetherian. 

\textbf{Solution.}
Let $\pi_i : A \to A / \fa_i$ denote the projections. We have the map 
\begin{equation*}
 \pi = (\pi_1, \dots, \pi_n): A \to A/\fa_1 \times \dots \times A/\fa_n.
\end{equation*}
As the $\fa_i$ have intersection $\{0\}$, $\pi$ is injective. Hence $A$ is isomorphic
to the subring $\img(\pi) \subset A/\fa_1 \times \dots \times A/\fa_n$. This 
shows that $A$ is isomorphic to the subring of a noetherian ring, thereby
noetherian.

\exercise{2}
Consider the matrix
\begin{equation*}
    S \coloneqq \begin{pmatrix}
        -36 & 14 & -24 \\
        18 & 6 & 12
    \end{pmatrix}.
\end{equation*}
Determine its elementary divisors and the kernel/cokernel of the map
$\Z^3 \xto S \Z^2$ (up to isomorphy).

\textbf{Solution.}  
We want to find simpler representatives of the residue class of $S$ in the double
quotient $\GL_2(A) \backslash \Mat_{2 \times 3}(A) / \GL_3(A)$. We add twice
the lower row to the upper row (which is the same as multiplying by $\smat 1201$ 
from the left), which gives
\begin{equation*}
    S \sim \begin{pmatrix}
        0 & 26 & 0 \\
        18 & 6 & 12
    \end{pmatrix}.
\end{equation*}
Further transformations yield
\begin{equation*}
    \begin{pmatrix}
        0 & 26 & 0 \\
        18 & 6 & 12
    \end{pmatrix}
    \to 
    \begin{pmatrix}
        0 & 26 & 0 \\
        6 & 6 & 12
    \end{pmatrix}
    \to
    \begin{pmatrix}
        0 & 26 & 0 \\
        6 & 0 & 0
    \end{pmatrix}
    \to 
\begin{pmatrix}
        6 & 0 & 0 \\
        0 & 26 & 0
    \end{pmatrix}.
\end{equation*}
This allows us to calculate kernel and cokernel of $S$. We find
\begin{equation*}
    \ker(S) \cong \Z, \quad \coker(S) \cong \oplus \Z/6\Z \oplus \Z/26\Z \cong
    \Z/2\Z \oplus \Z/78\Z.
\end{equation*}
This shows that the elementary divisors are given by $2$ and $78$. 

\exercise{3}
Let $A$ be a ring, let $\fa \subset A$ be an ideal and let $M, N_i$, $i \in I$, be $A$-modules for some set $I$. 
\begin{enumerate}
    \item Show that there exists a unique isomorphism
        \begin{equation*}
            \Phi: \bigoplus_{i \in I} (N_i \otimes_A M) \to \left( \bigoplus_{i
                \in I} N_i \right ) \otimes_A M
        \end{equation*}
        such that $\Phi((\dots,0, n_i \otimes m, 0 \dots)) = (\dots, 0, n_i, 0,\dots) \otimes m$ for all $n_i \in N_i$, $i \in I$, $m \in M$. 
    \item Show that there exists a unique isomorphism
        \begin{equation*}
            \Psi: A / \fa \otimes_A M \to M/\fa M
        \end{equation*}
        such that $\Psi((a + \fa) \otimes m) \mapsto am + \fa M$ for all 
        $a \in A$, $m \in M$. 
\end{enumerate}

\textbf{Solution.} 
This exercise looks like you'd have to do lots of calculations, but there is the
following rule:
\begin{equation*}
    \textit{NEVER DO ANYTHING EXPLICITLY WHEN WORKING WITH TENSOR PRODUCTS.}
\end{equation*}
\begin{enumerate}
    \item We could try to solve this by somehow checking that the map is
        well-defined, working everything out element-wise, and in the end
        showing that the isomorphism we obtain is somehow unique. But this is
        messy, and probably confusing to anyone who wants to follow the
        argument. It is much cleaner to work with universal properties. 
        Note that $\bigoplus_{i \in I} (N_i \otimes_A M)$ satisfies the
        following universal
        property: 
        \begin{center}
        \textit{For any $A$-module $P$ and any tuple of bilinear maps
        $(\phi_i: N_i \times M \to P)_i$,} \\ \textit{there is a unique linear map 
        $\Phi: \bigoplus_{i \in I} (N_i \otimes M) \to P$ such that 
        $\Phi(n_i \otimes m) = \phi_i(n_i, m)$.}
        \end{center}
        That $\bigoplus_{i \in I} (N_i \otimes_A M)$ satisfies this 
        universal property is easy to see. The UP of the tensor product
        gives linear maps $N_i \otimes M \to P$ associated to $\phi_i$, and
        we obtain $\phi$ by the UP of the direct sum. But note that 
        $(\bigoplus_{i \in I} N_i) \otimes M$ satisfies the same UP. Indeed, one 
        easily checks that a tuple of bilinear maps $(\phi_i: N_i \times M \to
        P)_{i \in I}$ is the same data as a single bilinear map $(\phi:
        (\bigoplus_{i \in I} N_i) \times M \to P)$. 
        This automatically gives a unique isomorphism 
        \begin{equation*}
            (\bigoplus_{i \in I} N_i) \otimes M   \cong
            \bigoplus_{i \in I} (N_i \otimes M),
        \end{equation*}
        which is of the desired form by construction.

    \item I lied to you, this time we do things explicitely. The mapping
            \begin{equation*}
                A/\fa \times M \to M / \fa M, \quad (a + \fa, m) \mapsto 
                am + \fa M.
            \end{equation*}
            is well-defined and bilinear, which is easy to check. This gives the
            desired map $\Psi: A/\fa \otimes_A M \to M/\fa M$. It is surjective 
            as $\Psi(1 \otimes m) = m + \fa M$, and injective because if 
            $\Psi((a + \fa) \otimes m) = 0 + \fa M$, we have $am \in \fa M$.
            Hence $am = a' m'$ for some $a' \in \fa, m' \in M$. In particular,
            $$a \otimes m = 1 \otimes (am) = 1 \otimes (a' m') = a' \otimes m' = 0
            \in A/\fa \otimes_A M.$$ 
            This shows injectivity of $\Psi$, and we are done.


            HAHAHA FOOLS! The proof above doesn't work! Namely, to show injectivity,
            it does not suffice to check that there are no nontrivial
            elements of the form $a \otimes m$ that get sent to zero. There might
            still be linear combinations of such elements which are getting
            sent to zero. But showing that $\sum a_i \otimes m_i \mapsto 0
            \implies \sum a_i \otimes m_i = 0$ is really hard, there is no way
            to get a handle on the sum. 
            
            So we try UPs again. We show that for any bilinear map
            $(-,-): A/\fa \times M \to P$ there is a unique linear map 
            $\phi: M/\fa M \to P$ with $\phi(am) = (a,m)$. This can be checked
            directly. 
\end{enumerate}

\exercise{4}
Let $A$ be a ring and let $M,N$ be $A$-modules. A bilinear map 
$(-,-): M \times M \to N$ is called symmetric if $(m_1, m_2) = (m_2, m_1)$ for all
$m_1, m_2 \in M$. It is called alternating if $(m,m) = 0$ for all $m \in M$. 
\begin{enumerate}
    \item Show that there exists an $A$-module $\Sym^2_A(M)$ and a symmetric bilinear
        map $\iota: M \times M \to \Sym^2_A(M)$ with the following universal
        property: For every $A$-module $N$ and for every symmetric bilinear map
        $(-,-): M \times M \to N$ there exists a unique $A$-linear map $\Phi:
        \Sym^2_A(M) \to N$ usch that for all $m_1, m_2 \in M$
        \begin{equation*}
            (m_1, m_2) = \Psi(\iota(m_1, m_2)).
        \end{equation*}
        Construct similarly an $A$-module $\Lambda^2_A(M)$ with a universal
        alternating bilinear map $\gamma: M \times M \to \Lambda^2_A(M)$. 

    \item Show that $\Sym^2_A(A^n)$ and $\Lambda^2_A(A^n)$ are free $A$-modules
        of ranks $\frac{n(n+1)}2$ and $\frac{n(n-1)}2$. 
\end{enumerate}

\textbf{Solution.}
\begin{enumerate}
    \item
        Okay, the $\Sym$-construction should be somehow similar to the construciton
        of $\otimes$, and ideally all proofs of properties simply follow from
        the universal
        property of the tensor product. In the construction of the tensor product,
        $(m_1, m_2)$ corresponds to the image of $\phi(m_1 \otimes m_2)$ for 
        some suitable morphism $\phi$. Imposing that $(m_1, m_2) = (m_2, m_1)$ 
        corresponds to the statement that in $\Sym^2_A$, any morphism 
        should send $(m_1 \otimes m_2 - m_2 \otimes m_1)$ to zero. 
        Building on this, we define $\Sym^2_A(M)$ as $(M \otimes_A M)/ G$, where
        $G$ is the $A$-module generated by elements of the form $(m_1 \otimes m_2 -
        m_2 \otimes m_1)$. We check that this works. With the notation of the 
        exercise, we first obtain a morphism $\psi: M \otimes_A M \to N$ by
        the UP of the tensor product.
        \[\begin{tikzcd}[ampersand replacement=\&]
        	{M \times M} \&\& {M \otimes M} \\
        	\& N \&\& {\Sym^2_A(M) \cong (M \otimes_A M)/G}
        	\arrow["{(-,-)}"', from=1-1, to=2-2]
        	\arrow["{(m_1,m_2) \mapsto m_1 \otimes m_2}", from=1-1, to=1-3]
        	\arrow["\psi", dashed, from=1-3, to=2-2]
        	\arrow[from=1-3, to=2-4]
        	\arrow["\Psi", dashed, from=2-4, to=2-2]
        \end{tikzcd}\]
        By construction, we have $G \subset \ker \psi$, so by the universal property
        of kernels, $\psi$ extends uniquely to a morphism $\Psi: \Sym^2_A(M)
        \cong (M \otimes_A M) / G \to N$. 

        We define $\Lambda^2_A(M)$ similarly, this time we define $G$ as
        submodule of 
        $M \otimes_A M$ generated by elements of the form $(m \otimes m)$. 

    \item We'll again first focus on $\Sym^2_A$. First of all, note that the set of
        bilinear maps $(-,-): A^n \times A^n \to N$ with values in an $A$-module $N$ 
        is the same as the set of matrices $(a_{ij})_{i,j = 1, \dots, n}$ 
        with $a_{ij} \in N$. The argument essentially comes from linear algebra; 
        we simply associate to $(-,-)$ the matrix $((e_i, e_j))_{ij}$. 
        Now, note that the subset of symmetric bilinear forms corresponds to those
        matrices with $a_{ij} = a_{ji}$. The set of these matrices has a
        natural structure of a free $A$-module of rank $\frac{n(n+1)}2$. We
        need to show that this number is equal to the rank of $\Sym^2_A$. 
        But for any $A$-module $N$, we have established the isomorphisms
        \begin{multline*}
            N^{\frac{n(n+1)}2} \cong \{M = (a_{ij})_{ij} \mid a_{ij} \in N \text{ and
            } a_{ij} = a_{ji} \}\\ \cong \operatorname{SymBiHom}(A^n, A^n; N)
            \cong \Hom_A(\Sym^2_A(A^2) , N).
        \end{multline*}
        Here, SymBiHom$(A^n, A^n; N)$ denotes the space of symmetric bilinear maps
        $A^n \times A^n \to N$. 

        The functor sending $N$ to $N^{\frac{n(n+1)}2}$ is represented by 
        $A^{\frac{n(n+1)}2}$. Hence, utilizing the Yoneda-lemma, we find that 
        $A^{\frac{n(n+1)}2} \cong \Sym^2_A(A^n)$. 

        For $\Lambda^2_A(A^n)$, we do exactly the same. The only thing that 
        changes is the set of matrices we look at, as this time we have
        isomorphisms
        \begin{equation*}
            \{M = (a_{ij})_{ij} \mid a_{ij} \in N \text{ and
            } a_{ij} = - a_{ji} \text{ and } a_{ii} = 0\} \cong 
            \operatorname{AltBiHom}_A(A^n, A^n, N).
        \end{equation*}
        The space of matrices is quickly seen to be isomorphic to
        $N^{\frac{n(n-1)}2}$. 
        



\end{enumerate}


\contactend
\end{document}
