\documentclass[a4paper,11pt]{article}
\pagenumbering{arabic}
\usepackage{../environment}
\usepackage{blindtext}
\usepackage{hyperref}
\usepackage{../quiver}

\begin{document}

\begin{center}
    \huge{Solutions to Sheet 9}
\end{center}

\exercise{1}
Assume that $d \in \Z$ is not a square. Determine all $x,y,z \in \Z$ with
$\gcd(x,y,z) = 1$ and $x^2 - dy^2 = z^2$. 

\textbf{Solution.} 
We do the same as in the lecture. First note that
\begin{equation*}
    L = \{(x,y,z) \mid x^2 - dy^2 = z^2\} \cong \{(x,y) \in \Q^2 \mid
    x^2 - dy^2 = 1\} = L'.
\end{equation*}
Just as in the lecture we try to simultaneously solve the equations
\begin{equation*}
\begin{aligned}
    x^2 - dy^2 &= 1 \\
    qx + q &= y
\end{aligned}
\end{equation*}
for $q \in \Q$.
Some calculations later we arrive at the unique non-trivial solution $(x,y) =
(\frac{1+dq^2}{1-dq^2}, \frac{2q}{1+dq^2}).$ Writing $q = \frac uv$ with
$(u,v) = 1$, we find that all solutions are of the form
\begin{equation*}
    (x,y,z) = \begin{cases}
        (v^2 + du^2, 2uv, v^2 - du^2), &\text{ if }v^2+ du^2 \text{ odd}\\
        (\frac{v^2 + du^2}2, uv, \frac{v^2 - du^2}2), &\text{ if }v^2+ du^2
        \text{ even.}
    \end{cases}
\end{equation*}

\exercise{2}
Let $k$ be an algebraically closed field and let $f(x) \in k[x]$ be a polynomial.
Determine the set $\spec(k[x,y]/(y^2 - f(x)))$ and the cardinality of all fibers
of the map
\begin{equation*}
    \spec(k[x,y]/(y^2 - f(x))) \to \spec(k[x])
\end{equation*}
that is induced by the $k$ algebra homomorphism $k[x] \to k[x,y]/(y^2-f(x))$, 
$x \mapsto x$. 

\textbf{Solution.}
We have seen that the prime ideals of $k[x,y]$ are those of the form 
$(x-a, y-b)$ for $a, b \in k$. The prime ideals of $k[x,y]/(y^2-f(x))$ are now 
those which contain $y^2 - f(x)$. 

There are two types of prime ideals in $k[x]$. Those of the form $(x-a)$
for $a \in k$ and the zero-ideal. Let $\pi: \spec(k[x,y]/(y^2-f(x))) \to
\spec(k[x])$ denote the morphism on spectra induced by the inclusion. We
calculate the fibers. On the \textit{special} fibers we find
\begin{equation*}
    \pi^{-1}((x-a)) = \spec(k[x,y]/(y^2-f(x)) \otimes_{k[x],\ x \mapsto a} k).
\end{equation*}
We can calculate the tensor product explicitely. We find
\begin{equation*}
    k[x,y]/(y^2 - f(x)) \otimes_{k[x]} k = k[x,y]/(y^2-f(x), x-a) = 
    k[y]/(y^2-f(a)).
\end{equation*}
And here we have
\begin{equation*}
    k[y]/(y^2-f(a)) = 
    \begin{cases}
        k^2, \quad &\text{if }f(a) \neq 0\\
        k[y]/(y^2), \quad &\text{if } f(a) = 0.
    \end{cases}
\end{equation*}
Hence the fibers either are given by two distinct "degree 1"-primes or by a 
single "degree 2"-prime.

At the \textit{generic} fiber we have 
\begin{equation*}
    \pi^{-1}((0)) = \spec(k[x,y]/(y^2-f(x)) \otimes_{k[x],\ x \mapsto x} k(x)).
\end{equation*}
Here the algebra calculates to
\begin{equation*}
    k[x,y]/(y^2-f(x)) \otimes_{k[x],\ x \mapsto x} k(x) = k(x)[y]/(y^2-f(x)).
\end{equation*}
Now
\begin{equation*}
    k(x)[y] / (y^2 - f(x)) \cong \begin{cases}
        k(x)[y]/y^2, \quad &\text{if } f(x) = 0 \\
        k(x)^2, \quad &\text{if } f(x) = g(x)^2 \neq 0\\
        k(x)[\sqrt {f(x)}], \quad &\text{otherwise}.
    \end{cases}
\end{equation*}
In the first case we have one prime ideal, in the second there are two, 
in the third there is one again. Note that in all cases, we are somehow
"degree 2". In all three cases, the algebras lying over the primes
are $k(x)$-algebras of dimension 2.

\textbf{Remarks.}
\begin{enumerate}
\item There is a neat formula to calculate tensor products, which 
        Konrad called \textit{Torsten's magic potion formula.} It is given by
        the following:
        \begin{multline*}
            k[y_1, \dots, y_m]/I \otimes_{\phi, k[x_1, \dots, x_n]} k[z_1, 
            \dots, z_l]/J \\ \cong k[y_1, \dots, y_m , z_1, \dots, z_l]/(I, J,
            \phi(x_1) - \psi(x_1), \dots, \phi(x_n) - \psi(x_n)).
        \end{multline*}

\item Also, there is a neat explanation why the fibers of a morphism on
spectra induced by a morphism $f:A \to B$ over $\fp \in \spec(A)$ can be
expressed in the form $\spec(B \otimes_A k(\fp))$.
We will need another description of $\spec(R)$, which is 
\begin{equation*}
    \spec(R) = \{ f: R \to K \}/\sim,
\end{equation*}
where $K$ are arbitrary fields and $(f_1: R \to K_1) \sim (f_2: R \to K_2)$ if
and only if there is some field $K'$ with morphisms $K_1 \to K'$, $K_2 \to K'$
such that $f_1 = f_2$ after applying those morphisms. The bijections are given
by sending $\fp \in \spec(R)$ to the morphism $R \to k(\fp)$ (in one
direction), and by sending $f$ to $\ker(f)$ (in the opposite direction).
With this description, a morphism of rings induces a morphism on spectra by
precomposition.
Remember the universal property of the tensor product of rings:
\[\begin{tikzcd}[ampersand replacement=\&]
	T \\
	\& {A \otimes_R B} \& B \\
	\& A \& R
	\arrow[from=3-3, to=3-2]
	\arrow[from=3-3, to=2-3]
	\arrow[from=3-2, to=2-2]
	\arrow[from=2-3, to=2-2]
	\arrow[curve={height=-6pt}, from=3-2, to=1-1]
	\arrow[curve={height=6pt}, from=2-3, to=1-1]
	\arrow["{\exists!}"', dashed, from=2-2, to=1-1]
\end{tikzcd}\]
That is, given two $R$-algebras $A$ and $B$ and $T$ a morphism $A \otimes_R B \to T$
is the same as $R$-algebra morphisms $A \to T$, $B \to T$ such that everything
commutes with the structure morphisms from $R$.

Now back to the fiber. We find 
\begin{equation*}
    \spec(f)^{-1}(A \to k(\fp)) = \{[g: B \to K] \mid g \circ f \sim (A \to k(\fp))\}
\end{equation*}
and the set on the right is exactly given by the set of morphisms $g$ such that
there are commutative squares
\[
\begin{tikzcd}[ampersand replacement=\&]
	B \& K \\
	A \& {k(\fp)}
	\arrow["f", from=2-1, to=1-1]
	\arrow[from=2-1, to=2-2]
	\arrow["g"', from=1-1, to=1-2]
	\arrow[from=2-2, to=1-2]
\end{tikzcd}
\]
up to equivalence, which is the same as $\spec(B \otimes_A k(\fp))$ by the 
universal property of the tensor product.

\end{enumerate}


\exercise{3}
Let $m, n \geq 1$ and let $\zeta_m = \ec^{2 \pi \ic / m} \in \C$ be a primitive
$m$-th root of unity. Set $G \coloneqq \langle \zeta_m \rangle \subset \C^\times$. 
We let $G$ act on $A \coloneqq \C[T_1, \dots, T_n]$ via $(g, f(T_1, \dots, T_n))
\mapsto g \cdot f \coloneqq f(gT_1, \dots, gT_n)$. 
\begin{enumerate}
    \item Determine the ring of invariante $A^G \coloneqq \{f \in A \mid 
        g \cdot f = f \text{ for all } g \in G\}$.
    \item Set $m = n = 2$. Find a presentation $A^G \cong \C[X_1, \dots, X_k]
        /(h_1, \dots, h_l)$. 
\end{enumerate}

\textbf{Solution.}
\begin{enumerate}
    \item We simply write down what happens. Let $f = \sum_{\mathbf i = (i_1, \dots, i_n) \in \N^n} a_{\mathbf i}T^{\mathbf i} \in \C[T_1, \dots, T_N]$. Now applying
        $\zeta_m$ gives 
        \begin{equation*}
            \zeta_m f = \sum_{k = 0}^\infty \zeta_m^{k}
            \sum_{\abs{\mathbf i}= k} a_{\mathbf i}T^{\mathbf i},
        \end{equation*}
        where $\abs{\mathbf i} = \sum_{j=1}^n i_j$. Now it is easy to see that 
        $\zeta_m f = f$ if and only if the only $a_{\mathbf i} = 0$ 
        whenever $m \nmid \abs {\mathbf i}$. 
    \item By the above, we find that $A^G = \C[T_1^2, T_1T_2, T_2^2]$. 
        This is also given by $\C[X,Y,Z]/(Y^2 - XZ) \eqqcolon B$. To see 
        that $A^G \cong B$, look at 
        $\C[X,Y,Z] \to \C[T_1, T_2]$, $X \mapsto T_1^2, Y \mapsto T_1T_2, 
        Z \mapsto T_2^2$. The kernel of this morphism contains
        $(Y^2 - XZ)$. Also, the image, $A^G$, has Krull-dimension at least $2$, 
        as we have the chain of prime ideals $0 \subset (T_1^2, T_1T_2)
        \subset (T_1^2, T_1T_2, T_2^2)$. By Krull's PID theorem, the dimension
        of $\C[X,Y,Z]/(Y^2-XZ)$ is two. Hence the kernel is generated by 
        $Y^2-XZ$, as any other generator would decrease dimension even more.
\end{enumerate}


\exercise{4}
Let $A$ be a ring and $M$ be a finitely generated $A$-module. Let $n \geq 1$
and let $f: A^n \to M$ be a surjection. Show that $K \coloneqq \ker(f)$
is finitely generated.

\textbf{Solution.}
As $M$ is finitely generated, there is a short exact sequence 
$0 \to Q \to A^m \to M \to 0$ with $Q$ finitely generated. 
Our situation is now the following.
\[\begin{tikzcd}[ampersand replacement=\&]
	0 \& Q \& {A^m} \& M \& 0 \\
	0 \& K \& {A^n} \& M \& 0
	\arrow[from=1-1, to=1-2]
	\arrow[from=1-2, to=1-3]
	\arrow["g", from=1-3, to=1-4]
	\arrow[from=1-4, to=1-5]
	\arrow["f", from=2-3, to=2-4]
	\arrow[from=2-4, to=2-5]
	\arrow[from=2-1, to=2-2]
	\arrow["\id"{description}, from=1-4, to=2-4]
	\arrow[from=2-2, to=2-3]
	\arrow["{\exists \alpha?}"{description}, dashed, from=1-3, to=2-3]
	\arrow["{\exists \beta?}"{description}, dashed, from=1-2, to=2-2]
\end{tikzcd} \]
We want to construct morphisms $\alpha$ and $\beta$ making the diagram above commute,
in the hope of being able to apply the snake lemma then. 
First, we construct $\alpha$. It suffices to find values for $\alpha(e_i)$.
We simply choose any $\alpha(e_i) \in f^{-1}(g(e_i))$. Now by the universal
property of kernels, we also get $\beta$. We want to show that $K$ is 
finitely generated. The snake lemma gives a short exact sequence
\begin{equation*}
    0 \to \coker \beta \to \coker \alpha \to 0.
\end{equation*}
Hence, $\coker \beta \cong \coker \alpha \cong A^n / \img(\alpha)$ is 
finitely generated. We also have the short exact sequence
\begin{equation*}
    0 \to \img(\beta) \to K \to \coker(\beta) \to 0.
\end{equation*}
As $\img(\beta)$ is finitely generated, we obtain that $K$ is finitely generated.
Indeed, let $(f_1, \dots, f_n)$ be generators of $\img(\beta)$ and 
$(g_1, \dots, g_m)$ be lifts of generators of $\coker(\beta) = K/\img(\beta)$. 
Now we have a diagram
\[\begin{tikzcd}[ampersand replacement=\&]
	0 \& {A^n} \& {A^{m+n}} \& {A^m} \& 0 \\
	0 \& {\img(\beta)} \& K \& {\coker(\beta)} \& 0
	\arrow[from=1-1, to=1-2]
	\arrow[from=1-2, to=1-3]
	\arrow[from=1-3, to=1-4]
	\arrow[from=1-4, to=1-5]
	\arrow[from=2-3, to=2-4]
	\arrow[from=2-4, to=2-5]
	\arrow[from=2-1, to=2-2]
	\arrow["{e_j \mapsto g_j}"{description}, from=1-4, to=2-4]
	\arrow[from=2-2, to=2-3]
	\arrow[from=1-3, to=2-3]
	\arrow["{e_i \mapsto f_i}"{description}, from=1-2, to=2-2]
\end{tikzcd}\]
from where we can use the snake lemma agin.


\contactend

\end{document}
