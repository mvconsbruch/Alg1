\documentclass[a4paper,11pt]{article}
\pagenumbering{arabic}
\usepackage{../environment}

\begin{document}

\begin{center}
    \huge{Solutions to Sheet 4}
\end{center}

\exercise{1}
Let $A$ be a ring.
\begin{enumerate}
    \item Assume that $f_n \in A\llbracket T \rrbracket$, $n \geq 0$, is a sequence
        of elements such that $f_n \in (T)^n$ for all $n \geq 0$. Show that 
        there exists a unique element $f \in A\llbracket T \rrbracket$ such that 
        $f - \sum_{k=0}^n f_k \in (T)^{n+1}$ for all $n \geq 0$. 
    \item Assume that $A$ is noetherian. Show that $A\llbracket T \rrbracket$ is
        noetherian.
\end{enumerate}

\textbf{Solution.}
\begin{enumerate}
    \item We can just write down $f$. We need to find
        coefficients $a_n$ such that $f = \sum_{n=0}^\infty a_n T^n$ satisfies
        $f - \sum f_k \in (T)^{n+1}$. Write 
        $f_k = \sum_{j=0}^k a_{kj} T^j + (T)^k$. One quickly verifies that 
        $a_n = \sum_{k=0}^n a_{kn}$ does the job. 
    \item Similar to the proof that the polynomial ring over a noetherian ring is
        noetherian, we let $I \subset A\llbracket T \rrbracket$ denote any ideal
        and denote by $I'$ the ideal of $A$ generated by the leading coefficients of 
        functions in $f$, namely $I' \coloneqq (a_d \mid f = a_d T^d +
        a_{d+1}T^{d+1} + \dots \in A\lbr T \rbr)$. As $A$ is noetherian, there is 
        a finite number of elements $f_1, \dots, f_n$ such that the leading
        (non-zero) coefficients of $f_i$ gerate $I'$. Upon multiplying with
        powerst of $T$, we may assume that all $f_i$ are of the form $f_i =
        a_{id}T^d + \dots$ with $a_{id} \neq 0$ for some some 
        suitable $d$. 

        Now we claim that any $g \in I \cap T^{d}$ also lies in $(f_1, \dots,
        f_n)$. Indeed, writing $g = b_{d}T^{d} + b_{d+1} T^{d + 1} + \dots$, 
        we find that $b_{d} \in I'$, so we can eliminate the term $b_d T^d$ from 
        $g$ without leaving $I \cap T^d$. But now $g' = g - b_d T^d \in I \cap 
        (T^{d+1})$. Upon repeatedly eliminating leading coefficients, we find
        $g \in (f_1, \dots, f_n)$. 

        To finish the argument, note that $A\lbr T \rbr/(T^d) \cong A[T]/(T^d)$
        is noetherian. Hence the image of $I$ in this quotient is finitely generated,
        by $(g_1, \dots, g_m)$, say. Choose lifts $(\Tilde g_1, \dots, \Tilde g_m)$. 
        Now, by construction, $I = (g_1, \dots, g_m, f_1, \dots, f_n)$. 
        

\end{enumerate}

\exercise{2}
\begin{enumerate}
    \item Let $A$ be the ring of power series in $\C\llbracket z \rrbracket$ with
        a positive radius of convergence. Show that $A$ is noetherian.
    \item Show that the ring of holomorphic functions is not noetherian. 
\end{enumerate}

\textbf{Solution.}
\begin{enumerate}
\item One can quickly verify that all ideals of $A$ are of the form 
    $(z^d)$. Indeed, every function that does not vanish at $0$ does not have 
    a root in some neighbourhood of $0$ (by the identity theorem), hence admits
    a holomorphic inverse there. This shows that the units in $A$ are given by
    $A \setminus (z)$. Now any non-unit is of the form $z^d u$ with $u$ invertible
    and $d \geq 1$. The claim follows. 

\item The hint commanded us to make use of the equation 
    $\sin(2x) = 2 \sin \cos(x)$. This shows that there is an infinite
    descending chain of ideals $(\sin(x)) \subset (\sin(x/2)) \subset (\sin(x/4))
    \subset \dots$. It is clear that this chain does not get stationary, by 
    looking at the real roots of those functions. 
\end{enumerate}


\exercise{3}
Let $n \geq 1$. For an $n \times n$ matrix $M$ over some ring $A$ denote by 
$\chi_M(T) = \det( T \cdot \Id - M)$ its characteristic polynomial.
\begin{enumerate}
    \item Let $A = \Z[a_{ij} \mid 1 \leq i,j \leq n]$ and $M \coloneqq (a_{ij})_{ij}
        \in \Mat_n(A)$. Show that $\chi_M(M) = 0$. 
    \item Deduce a general form of the theorem of Cayley-Hamilton: Let $A$ be a ring
        and let $M$ be any $n \times n$ matrix over $A$. Then $\chi_M(M) = 0$. 
\end{enumerate}
\textbf{Solution.}
\begin{enumerate}
    \item Since $A$ is integral, we can pass to the field of fractions of $A$. 
        Now the regular cayley hamilton applies. (Note that the calculation of 
        the determinant does not depend on whether we are in the field of 
        fractions or not).
    \item There is a surjective map $\pi: \Z[a \in A] \to A$ given by 
        $a \mapsto a$. By part 1 we find that $\chi_M(M)=0$ in $\Z[a \in A]$. 
        Now $0 = \pi(\chi_M(M)) = \chi_M(M)$. Done?
\end{enumerate}

\exercise{4}
Let $A$ be a principal ideal theorem.
\begin{enumerate}
    \item Let $a \in A\setminus \{0\}$ and let $\pi \in A$ be prime. Set 
        $B \coloneqq A/(a)$. For any $n \geq 0$ show that 
        \begin{equation*}
            \dim_{A/(\pi)} \pi^n B/\pi^{n+1}B = \begin{cases}
                0, &\text{ if }\nu_\pi(a) \leq n\\
                1, &\text{ if }\nu_\pi(a) \geq n+1.
            \end{cases}
        \end{equation*}
    \item Assume that $M = A^r \oplus A/(a_1) \oplus \dots \oplus A/(a_k)$,
        $N = A^s \oplus A/(b_1) \oplus \dots \oplus A/(b_l)$ with 
        $a_i, b_i \in A$ non zero and $a_1 \mid a_2 \mid \dots \mid a_k$,
        $b_1 \mid \dots \mid b_l$. Show that if $M \cong N$, then $r=s$, 
        $k = l$ and $(a_i) = (b_i)$ for all $i$. 
\end{enumerate}

\contactend
\end{document}
