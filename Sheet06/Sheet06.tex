\documentclass[a4paper,11pt]{article}
\pagenumbering{arabic}
\usepackage{../environment}
\usepackage{../quiver}

\begin{document}

\begin{center}
    \huge{Solutions to Sheet 6}
\end{center}

\exercise{1}
Let $A$ be a ring, $f \in A$ a non-zero divisor, $\fa = (f)$ and $\fb \subset A$
an ideal. Show that the natural map
\begin{equation*}
    \fa \otimes_A \fb \to \fa \cdot \fb, \quad a \otimes b \mapsto a\cdot b
\end{equation*}
is an isomorphism.

\textbf{Solution.} 
As $\fa = (f)$, we have an isomorphism $\phi: A \xto \sim \fa$, given by 
$a \mapsto fa$. Also note that $\phi|_\fb: \fb \to \fa \fb$ is an isomorphism.
Now we have the diagram
\[\begin{tikzcd}[ampersand replacement=\&]
	{\fa \otimes_A \fb} \&\& {\fa \fb} \\
	\\
	{A \otimes_A \fb} \&\& \fb
	\arrow["\sim", from=3-1, to=3-3]
	\arrow["{\phi|_\fb}", from=3-3, to=1-3]
	\arrow["{\phi \otimes \fb}", from=3-1, to=1-1],
\end{tikzcd}\]
where all arrows are isos, yielding an isomorphism $\fa \otimes_A \fb \to \fa \fb$. 

\exercise{2}
Let $A$ be a ring, let $I$ be a set and let $M, N_i, i \in I$ be $A$-modules. 
\begin{enumerate}
    \item Assume that $M$ is finitely generated (resp. finitely presented). Show that the natural map 
        \begin{equation*}
        M \otimes_A \prod_{i \in I} N_i \to \prod_{i \in I} M \otimes_A N_i
        \end{equation*}
        is surjective (resp. bijective). 
    \item Take $A = \Z[X_0, X_1,  \dots]$, $J = (X_0, X_1, \dots)$. Show that the
        natural map $A/J \otimes_A A\lbr T \rbr \to A/J\lbr T\rbr$ is not
        injective.
\end{enumerate}
\textbf{Solution.}
\begin{enumerate}
    \item First let's recall what finitely generated and finitely presented meant.
        An $A$-module $M$ is finitely generated if there exists a surjective 
        morphism of $A$-modules
        \begin{equation*}
            A^{\oplus n} \to M.
        \end{equation*}
        Furthermore, we call $M$ finitely presented if the kernel of this map is
        again finitely generated (that is, there is a finite number of relations
        among the images of the generators), which is to say that there is an 
        exact sequence
        \begin{equation*}
            A^m \to A^n \to M \to 0
        \end{equation*}
        for some integers $m,n \geq 0$. 

        Next, let's find out what the \textit{natural map} is. We have for $i
        \in I$ the projections $\prod_{i \in I} N_i \to N_i$, which we can
        tensor with $M$ to obtain maps $M \otimes_A \prod_{i \in I} N_i 
        \to M \otimes_A N_i$. The collection of these maps gives the desired
        $M \otimes_A \prod_{i \in I} N_i \to \prod_{i \in I} M \otimes_A N_i$. 

        Note that if $M \cong A^{\oplus n}$, this natural map is an isomorphism,
        as we have
        \begin{equation*}
            A^{\oplus n} \otimes_A \prod_i N_i \cong (A \otimes_A \prod_i
            N_i)^{\oplus n} \cong (\prod_i N_i)^{\oplus n} \cong \prod_i
            (A^{\oplus n} \otimes_A N_i).
        \end{equation*}
        Here we used the commutativity of finite direct sums and tensor products and 
        that of finite direct sums and products (note that finite sums are isomorphic
        to finite products).

        This puts us in the following situation, where we can use the $5$-lemma.
        \[\begin{tikzcd}[ampersand replacement=\&]
            {A^{\oplus m} \otimes_A \prod_{i \in I} N_i} \& {A^{\oplus n}
            \otimes_A \prod_{i \in I} N_i} \& {M \otimes_A \prod_{i \in I} N_i}
            \& 0 \& 0 \\
            {\prod_{i \in I} (A^{\oplus m} \otimes_A  N_i)} \& {\prod_{i \in
            I}( A^{\oplus n} \otimes_A  N_i )} \& {\prod_{i \in I}( M \otimes_A
        N_i)} \& 0 \& 0.
        	\arrow[from=1-1, to=1-2]
        	\arrow[from=1-2, to=1-3]
        	\arrow[from=1-3, to=1-4]
        	\arrow["\sim", from=1-1, to=2-1]
        	\arrow[from=2-1, to=2-2]
        	\arrow[from=2-2, to=2-3]
        	\arrow[from=2-3, to=2-4]
        	\arrow[Rightarrow, no head, from=1-4, to=2-4]
        	\arrow[Rightarrow, no head, from=1-5, to=2-5]
        	\arrow[from=1-4, to=1-5]
        	\arrow[from=2-4, to=2-5]
        	\arrow["{\therefore \ \sim}", from=1-3, to=2-3]
        	\arrow["\sim", from=1-2, to=2-2]
        \end{tikzcd}\]

    \item Note that $A/J \otimes_A A\lbr T \rbr \cong A \lbr T \rbr / J A\lbr T 
        \rbr$. Take the element $f \coloneqq \sum_{i=1}^\infty x_i T^i \in
        A\lbr T \rbr$. As all elements in $J A\lbr T \rbr$ only have a finite
        number of $x_i$ arise in the coefficients, we find $f \not \in 
        J A \lbr T \rbr$, hence $f \neq 0$ in $A\lbr T \rbr / J A\lbr T \rbr$. 
        But $f \mapsto 0$ under the natural map: All the coefficients $x_i$
        get sent to zero.

\end{enumerate}

\exercise{3}
Let $k$ be a field, $K/ k$ an algebraic field extension, and $\bar k$ an alebraic
closure of $k$. 
\begin{enumerate}
    \item If $V \to W$ is a $k$-linear injection of $k$-vector spaces, show that
        $V \otimes \bar k \to W \otimes \bar k$ is a $\bar k$-linear injection.
    \item Show that $K/k$ is separable if and only if the ring $K \otimes_k \bar k$
        is reduced. 
\end{enumerate}
\textbf{Solution.}

\begin{enumerate}
    \item All $k$-vector spaces are injective, hence every injective map 
        $V \to W$ admits a section $W \to V$. Tensoring the section with 
        $\bar k$ yields a section of $V \otimes_k \bar k \to W \otimes_k \bar k$. 
    \item We show that the following statements are equivalent:
        \begin{enumerate}
            \item $K/k$ is separable.
            \item For all $\alpha \in K$, $k[\alpha]/k$ is separable.
            \item $k[\alpha] \otimes_k \bar k$ is reduced for all $\alpha \in K$.
            \item $K \otimes_k \bar k$ is reduced.
        \end{enumerate}
        (a) $\iff$ (b) is by definition. We show (b) $\iff$ (c). 
        Let $f$ be the minimal polynomial of some $\alpha \in K$. 
        As $K$ is algebraic over $k$, $f$ decomposes in $\bar k$ as 
        $f(x) = \prod_{i = 1}^n (x - a_i)^{d_i}$ with $a_i \neq a_j$ whenever
        $i \neq j$. Now we find 
        \begin{equation*}
            k[\alpha] \otimes_k \bar k \cong (k[x]/f(x)) \otimes_k \bar k
            \cong \bar k[x]/f(x) \cong k[x]/(x-a_1)^{d_1} \times \dots \times k[x] /
            (x-a_n)^{d_n},
        \end{equation*}
        which is reduced if and only if $d_1 = \dots = d_n = 1$, which is 
        the case if and only if $k[\alpha]$ is reduced over $k$.

        For (d) $\implies$ (c), we use part 1. The arguments there show that 
        $k[\alpha] \to K$ is injective, hence $k[\alpha] \otimes_k \bar k$ is
        isomorphic to a subring of $K \otimes_k \bar k$. But a subring of a
        reduced subring is reduced.

        Lastly we show (a) $\implies$ (d). Let $\zeta = \sum_{i=1}^n \alpha_i \otimes
        b_i \in K \otimes_k \bar k$ be some element. Here, the $\alpha_i$ 
        are elements of $K$, and as $K$ is separated, we find that 
        $k[\alpha_1, \dots, \alpha_n]/k$ is a finite separated extension. But
        now, by the primitive element theorem, there is some $\alpha \in K$
        with $k[\alpha] \cong k[\alpha_1, \dots, \alpha_n]$, and
        $k[\alpha] \otimes_k \bar k$ is reduced by (c) $\iff$ (a).
\end{enumerate}

\exercise{4}
Let $A$ be a ring and let $I$ be an \textit{invertible} $A$-module, i.e., 
there exists an $A$-module $J$ such that $I \otimes_A J \cong A$. Let
$\phi: M \to N$ be a homomorphism of $A$-modules. 
\begin{enumerate}
    \item Show that $\phi$ is nonzero (resp. injective, resp. surjective) 
        if and only if $\phi \otimes_A I: M \otimes_A I \to N \otimes_A I$
        is so. 
    \item Show that $I$ is finitely generated.
\end{enumerate}

\textbf{Solution.} 
\begin{enumerate}
    \item We have seen in the lecture that tensor products preserve surjectivity. 

        To see that $\phi = 0$ if and only if $\phi \otimes_A I = 0$, just tensor
        with $J$. 

        Lastly, suppose that $\phi \otimes_A I$ is injective. Let $\psi: K \to
        M$ be the kernel of $\phi$. We need to show that $\psi = 0$. 
        We are in the following situation:
        \[
            \begin{tikzcd}[ampersand replacement=\&]
            	K \& M \& N \&\& {I \otimes K} \& {I \otimes M} \& {I \otimes N}
            	\arrow["\phi", from=1-2, to=1-3]
            	\arrow["\psi", from=1-1, to=1-2]
            	\arrow["0"', curve={height=12pt}, from=1-1, to=1-3]
            	\arrow["{\psi \otimes I}", from=1-5, to=1-6]
            	\arrow["{\phi \otimes I}", from=1-6, to=1-7]
            	\arrow["0"', curve={height=12pt}, from=1-5, to=1-7]
            \end{tikzcd}
        \]
        We know that $\phi \otimes I$ is injective, hence $\psi \otimes I$ has 
        to be zero. But by preserving $0$, this shows that $\psi = 0$, hence
        the kernel of $\phi$ vanishes. This shows that $\phi$ is injective.
        The same argument replaced with $J$ shows that $\phi$ is injective if 
        $\phi \otimes I$ is.

    \item We have an isomorphism $\phi: I \otimes_A J \cong A$. Let's look at
        the preimage of $1$ under $\phi$. It is given by some finite sum
        $\phi^{-1}(1) = \sum_{k = 1}^n i_k \otimes j_k$. We claim that 
        $i_1, \dots, i_n$ generate $I$. Indeed, look at the morphism 
        $\psi: A^n \to I$, $e_k \mapsto i_k$. Upon tensoring with $J_k$ we obtain
        a morphism $\psi \otimes_A J: J^n \to A$, and $1 \in A$ lies in
        the image. Hence $\psi \otimes_A J$ is surjective. But this shows that
        $\psi$ is surjective (by part 1).
\end{enumerate}


\contactend
\end{document}
