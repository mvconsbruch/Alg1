\documentclass[a4paper,11pt]{article}
\pagenumbering{arabic}
\usepackage{../environment}

\begin{document}

\begin{center}
    \huge{Solutions to Sheet 3}
\end{center}

\exercise{1}
Let $A$ be a PID. The arguments of $A = \Z$ from the lecture work 
verbatim to show that the prime ideals of $A[T]$ are 
\begin{enumerate}
    \item $(0)$,
    \item $(f)$, $f \in A[T]$ irreducible,
    \item $(\pi, g)$ with $\pi \in A$ prime and $g \in A[T]$ a polynomial whose
        image in $(A/\pi)[T]$ is irreducible.
\end{enumerate}
Show the following.
\begin{enumerate}
    \item Assume that $A$ has infinitely many prime ideals. Prove that the
        heights of the primes in (i), (ii) and (iii) are given by 
        $0,1,2$ respectively. Show that each maximal ideal of $A[T]$ has height $2$. 
    \item Let $k$ be a field and set $A = k\llbracket u \rrbracket$. Show that 
        $A[u^{-1}]$ is a field. Deduce that, in contrast to 1), the height $1$
        ideal $(uT - 1)$ is maximal. 
\end{enumerate}
\textbf{Solution.}
\begin{enumerate}
    \item That $(0)$ is of height zero is obvious. We showed on the last sheet
        that the only prime ideals contained in principal prime ideals 
        $(f)$ of UFDs are $(0)$ and $(f)$. As polynomial rings over UFDs are 
        UFDs again, we are done with this case. 

        To solve the first part of the exercise, we show that every ideal of the
        form $(f)$ is contained in some ideal of the form $(\pi, g)$, and that
        there are no nontrivial inclusions of ideals that are of that form.

        Assume we are given two prime ideals $ \fp = (\pi, g) \subset (\pi',
        g') = \fp'$. By this inclusion we find $\fp \cap A = \fp' \cap A$, 
        which shows $(\pi) = (\pi')$. But 
        $A/(\pi)$ is a field, hence $A/(\pi)[T]$ is a PID and we find that 
        the reductions of $g$ and $g'$ mod $\pi$ are the same. This shows 
        $(\pi, g) = (\pi, g')$, and we are done. (We have not used yet that 
        there are infinitely many prime ideals).

        We also have to show that every maximal ideal is of the form $(\pi, g)$. 
        To this end, we have to show that every ideal of the form $(f)$ is
        contained in some ideal $(\pi, g)$. But if we write such $f$ 
        as $f = a_d T^d + \dots + a_0$ and choose some prime $\pi \in A$ 
        that does not divide $a_d$, we find that the reduction of $f$ mod $\pi$
        is monic, at least up to multiplication with some unit. Hence we can 
        choose some irreducible factor $g \in (A/\pi)[T]$ of $f$ and lift
        it to a function $\tilde g \in A[T]$. We find that $(\pi, \tilde g)$ 
        is prime and contains $(f)$, as desired. 
        
    \item We have $A[T]/(uT-1) = A[u^{-1}]$. Note that $A$ is a local ring and in
        particular a principal ideal domain (but with only a single prime). 
        We have seen on a prior sheet that every element $x \in A \setminus (u)$
        is invertible, hence we done. 
\end{enumerate}


\exercise{2}
Let $k$ be an algebraically closed field and let 
\begin{equation*}
    \phi: k[x,y] \to k[u,v], \quad x \mapsto u, \ y \mapsto uv.
\end{equation*}
\begin{enumerate}
    \item Use exercise 1 to show that the maximal ideals of $k[x,y]$ are precisely
        the ideals 
        \begin{equation*}
            \fm_{\lambda, \mu} \coloneqq (x-\lambda, y - \mu), \quad \lambda,
            \mu \in k.
        \end{equation*}
    \item Show that $\phi$ induces an isomorphism $k[x,y][x^{-1}] \to
        k[u,v][u^{-1}]$. 
    \item For each $(\lambda, \mu) \in k^2$ calculate
        $\spec(\phi)^{-1}(\fm_{\lambda, \mu})$. 
\end{enumerate}

\textbf{Solution.}
\begin{enumerate}
    \item By exercise 1, the maximal ideals are precisely the ideals of the 
        form $(\pi, g)$ where $\pi \in k[x]$ is prime and $g \in k[x,y]$ is an
        element with reduction mod $\pi$ is irreducible. As $k$ is algberically
        closed, we find that $(\pi) = (x-\lambda)$ for some $\lambda \in k$.
        Now $k[x]/(x-\lambda) \cong k$, the isomorphism is given by $x \mapsto 
        \lambda$. Hence $g(\lambda, y) \in k[y]$ needs to be irreducible, 
        i.e., of the form $y - \mu$. 
    \item We can give an isomorphism $k[u,v][1/v] \to k[x,y][1/x]$ via
        $u \mapsto x$ and $v \mapsto y/x$. Checking that this is an isomorphism
        is straight-forward. 
    \item Note that $\spec(\pi)^{-1}(\fm_{\lambda, \mu})$ is equal to the set of
        prime ideals $\fp \subset k[u,v]$ for which $\phi(\fm_{\lambda, \mu})
        \subset \fp$. By the homomorphism theorem, we find
        \begin{equation*}
            \{\fp \subset k[u,v] \text{ prime} \mid \phi(\fm_{\lambda, \mu})
            \subset \fp\} \xto{1:1} \spec(k[u,v]/\phi(\fm_{\lambda, \mu})).
        \end{equation*}
        We find $\phi(\fm_{\lambda, \mu}) = (u - \lambda, uv - \mu)$, hence 
        $$k[u,v] / \phi(\fm_{\lambda, \mu}) \cong k[u,v]/(u - \lambda, uv - \mu)
        \cong k[v]/(v \lambda - \mu).$$
        But this can be calculated explicitely:
        \begin{equation*}
            k[v]/(v \lambda - \mu) \cong \begin{cases}
                k, &\text{ if } \lambda \neq 0 \\
                k[v], &\text{ if } \lambda = 0 \text{ and } \mu = 0\\
                0, &\text{ if } \lambda = 0 \text{ and } \mu \neq 0\\
            \end{cases}
        \end{equation*}
\end{enumerate}

\exercise{3}
    Let $A$ be a ring of Krull dimension $n \coloneqq \dim A$. Show that 
    \begin{equation*}
        n+1 \leq \dim A[T] \leq 2n + 1.
    \end{equation*}

\textbf{Solution.}
Let $$0 = \fq_0 \subset \fq_1 \subset \dots \subset \fq_m$$ be a maximal chain of 
prime ideals in $A[T]$. Write $\fp_i = \fq_i \cap A$. We obtain an ascending chain
of prime ideals 
\begin{equation*}
    0 = \fp_0 \subset \fp_1 \subset \dots \subset \fp_m
\end{equation*}
in $A$, which by the assumption on the dimension of $A$ contains at most $n+1$
different prime ideals. We will show that $\fp_i = \fp_{i+1}$ implies
$\fp_{i+1} \neq \fp_{i+2}$, which shows $m \leq 2n + 1$. 

More generally, we'll show that for every prime ideal $\fp \subset A$ the set of
prime ideals $\fq \subset A[T]$ with $\fq \cap A = \fp$ (the primes above
$\fp$) can have chains of length at most two. Let $\fp \subset A$ be such a prime
ideal. By passing to the localization $A_\fp = S^{-1} A$ with $S = A\setminus \fp$,
we may assume that $\fp$ is maximal in $A$. 
Now $A/\fp$ is a field, hence $A/\fp[T]$ is a PID, hence of dimension $1$. 
The inclusion-preserving bijection
\begin{equation*}
    \{\fq \subset A/\fp[T]\} \xto{1:1} \{\fq \subset A[T] \mid A[T]\fp \subset
    \fq \}
\end{equation*}
solves the exercise.

\exercise{4}
Let $A$ be a ring and $S, T \subset A$ multiplicative subsets with $S \subset T$. 
\begin{enumerate}
    \item Let $\iota_S : A \to S^{-1}A$ be the natural ring homomorphism. Show that 
        $\iota_S^{-1}((S^{-1} A)^\times)$ is the saturation $\bar S$ of $S$. 
    \item Show that there exists a unique ring homomorphism $\iota:  S^{-1}A \to
        T^{-1}A$ such that $\iota \circ \iota_S = \iota_T.$
    \item Deduce that $\iota$ is an isomorphism if and only if $\bar S = \bar T$. 
\end{enumerate}

\textbf{Solution.}

First a reminder: The saturation of a subset $S \subset A$ is given by the set
$$\bar S = \{s \in A \mid \exists a \in A : as \in S\}.$$

\begin{enumerate}
    \item First remember that $\iota_S$ is given by $a \mapsto \frac a1$. 
        Now let's try to work out what the units in $S^{-1}A$ are. Remember that
        $S^{-1}A$ has underlying set 
        \begin{equation*}
            (A \times S) / \sim_S, \quad \text{where} \quad
            (a,s) \sim_S (a', s') \text{ iff } as' = a's. 
        \end{equation*}
        In particular, we find that an element $(a,1)$ ($=\frac a1 = \iota_S(a)$)
        lies in the units of
        $S^{-1}A$ if and only if there are $a' \in A$, $s' \in S$ with 
        $(a a', s') \sim_S (1,1)$. This condition is equivalent to
        $(a a', 1) \sim_S (s', 1)$, which translates directly to what we had to
        show. 

    \item We define $\iota$ on representing objects using the inclusion
        $A \times S \to A \times T$. It is clear that this morphism behaves well
        under the equivalence relations $\sim_S$ and $\sim_T$ (here $\sim_T$ is
        defined the same way as for $\sim_S$), so we obtain a well-defined function
        \begin{equation*}
            S^{-1}A \cong (A \times S)/\sim_S \to (A\times T)/\sim_T \cong T^{-1}A.
        \end{equation*}
        One readily checks that this indeed gives a map of rings (with
        addition and multiplication defined accordingly). One also readily checks 
        that $\iota \circ \iota_S = \iota_T$. 

    \item We show that the saturation of $S$ is maximal among the subsets $S
        \subset S' \subset A$ with $S'^{-1} A \cong S^{-1} A$ (where the induced 
        morphism is given by $\iota$). 
        We first note that there is no difference between localizing at $S$ and
        localizing at $\bar S$. Indeed, given some $s \in \bar S$, there is 
        some $a \in A$ with $as \in S$. But now, given any $b \in A$, the element 
        $\frac bs \in \bar S^{-1} A$ lies in the same equivalence class
        as $\frac{ba}{sa} \in S^{-1} A$. (Alternatively, this follows directly from 
        what we showed in part 1: We have $\bar S^{-1} A = \bar S^{-1}(S^{-1} A) = S^{-1}A$,
        where we used in the last equality that $\bar S \subset
        (S^{-1}A)^\times$). Next, any subset $S \subset T \subset A$
        that is not contained in $\bar S$ has non-isomorphic localization.
        Indeed, assume $t \in T \setminus \bar S$. Then the
        equivalence class of $\frac 1t \in T^{-1}A$ does not lie in the image
        of $\iota$ by construction. Finally, note that whenever
        $S \subset T \subset \bar S$, we have $\bar T = \bar S$. 
        
        This solves the exercise in an instant. For the one direction, if
        $\bar S = \bar T$, we find 
        \begin{equation*}
            S^{-1} A \cong \bar S^{-1} A = \bar T^{-1}A \cong T^{-1}A.
        \end{equation*}
        For the other direction, if $S^{-1} A \cong T^{-1} A$, the result above
        directly implies $\bar S = \bar T$. 
\end{enumerate}


\contactend
\end{document}
