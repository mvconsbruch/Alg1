\documentclass[a4paper,11pt]{article}
\pagenumbering{arabic}
\usepackage{../environment}
\usepackage{blindtext}
\usepackage{hyperref}
\usepackage{../quiver}

\begin{document}

\begin{center}
    \huge{Solutions to Sheet 8}
\end{center}

\exercise{1}
Let $A$ be a ring and $\fa \subset A$ an ideal. Show that $A/\fa$ is finitely 
presented if and only if $\fa$ is a finitely generated ideal.

\textbf{Solution.} 
Remember that an $A$-algebra $B$ is of finite presentation iff there is an isomorphism
$A[X_1, \dots, X_n]/(f_1, \dots, f_r) \cong B$ with $f_i \in A[X_1, \dots,
X_n]$. If $\fa$ is finitely generated, clearly $A/\fa$ is of finite
presentation. Now suppose that $A/\fa \cong A[X_1, \dots, X_n]/(f_1, \dots, f_r)$. 
We have the following diagram with the horizontals being short exact sequences:
\[
\begin{tikzcd}[ampersand replacement=\&]
	0 \& {(f_1, \dots, f_r)} \& {A[X_1, \dots, X_n]} \& {A[X_1, \dots, X_n]/(f_1, \dots, f_r)} \& 0 \\
	0 \& \fa \& A \& {A/\fa} \& 0
	\arrow[from=1-2, to=1-3]
	\arrow[from=1-3, to=1-4]
	\arrow[from=1-1, to=1-2]
	\arrow[from=1-4, to=1-5]
	\arrow[from=2-1, to=2-2]
	\arrow[from=2-2, to=2-3]
	\arrow[from=2-3, to=2-4]
	\arrow["\cong", from=1-4, to=2-4]
	\arrow["\beta"', two heads, from=1-3, to=2-3]
	\arrow["\alpha"', from=1-2, to=2-2]
	\arrow[from=2-4, to=2-5]
	\arrow[Rightarrow, no head, from=1-5, to=2-5]
	\arrow[Rightarrow, no head, from=1-1, to=2-1]
	\arrow["\gamma"', from=1-4, to=2-4]
\end{tikzcd}
\]
Here, the map $\beta$ exists because every map is also a morphism of $A$-algebras,
and in particular send $1$ to $1$. Now $\alpha(f_i)$ is defined by the image
of $f_i$ in $A$, which lies in $\fa$ as $\alpha(f_i) = 0$ after projection
to $A/\fa$ (by commutativity of the diagram).
We need to show that $\alpha$ is surjective. There are many ways to see this, for example
we can use functoriality of kernels and the fact that $\beta$ splits, or we can use
the snake lemma, or simply do a diagram chase. 

\exercise{2}
Let $k$ be a field. Show that the ring extensions $k[X+Y] \to k[X,Y]/(XY)$ and
$k[X^2-1] \to k[X]$ are integral.

\textbf{Solution.} 
\begin{enumerate}
    \item Let $f(T) = T^2-T(X+Y)$. Then $f(X) = X^2 - X(X+Y) = -XY = 0$ in $k[X,Y]/(XY)$.
    \item Let $f(T) = T^2-1 - (X^2-1)$. Then $f(X) = 0$.
\end{enumerate}
In both cases, the extension is generated by elements for which we found monic polynomials
that have those elements as roots, hence they are generated by algebraic elements, 
hence algebraic.

\exercise{3}
Let $\phi: A \to B$ be a finite morphism of rings, i.e., $A \to B$ is a ring homomorphism
which makes $B$ a finite $A$-module. Show that the map $\spec(B)\to\spec(A)$
has finite fibers.

\textbf{Solution.} 
First, we find out what the fiber above a prime $\fp \in \spec(A)$ is. Writing down definitions, we find that it's given by
it is given by 
\begin{equation*}
    \{\fq \in \spec(B) \mid \phi^{-1}(\fq) = \fp\} = \{\fq \in \spec(B) \mid
    \phi(\fp) \subset \fq \subset \phi(\fp') \  \forall \fp' \supset \fp \}
\end{equation*}
By the homomorphism theorems, this is given by $\spec( B \otimes_A \kappa(\fp) )$.
But as $B$ is a finite $A$-module, there is a surjection (of $A$-modules) $A^n
\to B$, which turns into
a surjection (of $\kappa(\fp)$-vector spaces) $\kappa(\fp)^n \to B \otimes_A
\kappa(\fp)$. Hence $B \otimes_A \kappa(\fp)
\eqqcolon B_{\kappa(\fp)}$
is a finite $\kappa(\fp)$-algebra (in the sense that it is finitely generated as an
$A$-module). We now use ideas from Sheet 1, exercise 4. First, note that every prime
in $B_{\kappa(\fp)}$ is maximal, because every finite integral extension of a field
is a field. Now given any set $\{\fm_1, \dots, \fm_N\}$ of prime (hence maximal) ideals
we have an isomorphism
\begin{equation*}
    B_{\kappa(\fp)}/(\fm_1 \cap \dots \cap \fm_N) \xto \sim B_{\kappa(\fp)}/\fm_1 \times \dots \times B_{\kappa(\fp)}/\fm_N.
\end{equation*}
The object on the left has $\kappa(\fp)$-dimension $\leq \dim_{\kappa(\fp)}
B_{\kappa(\fp)}$, and the object on the right has $\kappa(\fp)$-dimension 
$\geq N$. In particular, there aren't more that $\dim_{\kappa(\fp)} B_{\kappa(\fp)}$ prime ideals
in $B_{\kappa(\fp)}$. 

\exercise{4}
Prove the $5$-lemma.

\textbf{Solution.}
I don't want to prove the $5$-lemma. I feel like the proof is a bit
involved and you don't get much insight from proving something so elementary.
However, I also don't want to discourage you from reading up the proof if you
feel like it! There are many proofs of this statement in various levels of
complicatedness and generality. If you are just interested in how to prove the
$5$-lemma for modules over rings (as in the exercise), you can simply do a
diagram chase. This has been done (for example) on
Wikipedia.\footnote{\url{https://en.wikipedia.org/wiki/Five_lemma}}
There is a proof making extensive use of the snake
lemma\footnote{\href{https://math.stackexchange.com/questions/2508678/proving-the-strong-four-lemma-using-the-snake-lemma}{https://math.stackexchange.com/...}},
and this even generalizes to arbitrary abelian categories.
The most elegant proof I know of uses spectral sequences. It is stated as 
exercise 1.7.C in Ravi Vakil's
book.\footnote{\url{https://math.stanford.edu/~vakil/216blog/FOAGnov1817public.pdf}
} ((I love this book))

\contactend

\end{document}
