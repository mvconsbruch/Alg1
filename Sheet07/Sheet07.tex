\documentclass[a4paper,11pt]{article}
\pagenumbering{arabic}
\usepackage{../environment}
\usepackage{../quiver}

\begin{document}

\begin{center}
    \huge{Solutions to Sheet 7}
\end{center}

\exercise{1}
Let $A \to B$ be a homomorphism of rings, let $M$ be an $A$-module and 
let $N$ be a $B$-module. 
\begin{enumerate}
    \item Show that the map
        \begin{equation*}
            \Hom_A(M,N) \to \Hom_B(B \otimes_A M, N), \quad
            \phi \mapsto (b \otimes m \mapsto b \phi(m))
        \end{equation*}
        is a well-defined isomorphism.

    \item Show that the map
        \begin{equation*}
            M \otimes_A N \to (M \otimes_A B) \otimes_B N, \quad
            m \otimes n \mapsto (m \otimes 1) \otimes n
        \end{equation*}
        is a well-defined isomorphism. 

    \item Deduce that $S^{-1} M_1 \otimes_A S^{-1}M_2 \cong
        S^{-1} M_1 \otimes_{S^{-1}A} S^{-1} M_2$
        for two $A$-modules $M_1, M_2$ and a multiplicative subset 
        $S \subset A$. 
\end{enumerate}
\textbf{Solution.}
\begin{enumerate}
    \item A function $\phi \in \Hom_B(B \otimes_A M , N)$ is uniquely determined
        by its values on elementary tensors. We have $\phi(b \otimes m) = 
        b \phi(1 \otimes m)$, so in reality $\phi$ is uniquely determined by 
        its values on $1 \otimes m$. But any such morphism gives rise to 
        a $A$-linear map via $m \mapsto 1\otimes m \mapsto \phi(1 \otimes m)$, 
        and conversely any $\psi \in \Hom_A(M,N)$ yields a unique morphism
        via $b \otimes m \mapsto b \psi(m) \in \Hom_B(B \otimes_A M, N)$. 
        These constructions are quickly checked to be mutually inverse.

        \textbf{Remark.} This is a special case of the so called 
        \textit{Hom-Tensor adjunction}. It states that there is a 
        natural isomorphism
        \begin{equation*}
            \Hom_B(M \otimes_A L, N) \cong \Hom_A(M, \Hom_B(L,N)).
        \end{equation*}
        In more fany terms, this says that the functors
        $\Hom_B(L, -): \Mod_B \to \Mod_A$ and $- \otimes_A L: \Mod_A
        \to \Mod_B$ is an adjoint pair, for any $B$-module $L$.

    \item Again, universal properties. Of course, we'll want to show that this
        is an isomorphism of $B$-modules. We do this by using the universal property.
        What is a $B$-linear morphism $\phi: (M \otimes_A B) \otimes_B N 
        \to P$? The same as a $B$-bilinear map $\Phi: (M \otimes_A B) \times N 
        \to P$. But as $\Phi(m \otimes b, n) = b \Phi(m \otimes 1, n)$, any
        such bilinear map is uniquely determined by its values on elements of 
        the form $(m \otimes 1 , n)$, hence it really is the same as a
        $A$-bilinear map $M \times N \to P$, given by
        $(m,n) \mapsto (m \otimes 1, n) \mapsto \Phi(m \otimes 1, n)$.
        This construction is quickly verified to be an isomorphism. But now
        $(M \otimes_A B) \otimes_B N$ satisfies the universal property of $M
        \otimes_A N$. 

    \item We apply the above with $S^{-1}M_1 = M$ and $S^{-1}M_2 = N$ and
        $B = S^{-1}A$. Note that $S^{-1}M_1 \otimes_A S^{-1}A \cong 
        S^{-1}(S^{-1}M_1) \cong S^{-1}M_1$, which gives (following the above)
        \begin{equation*}
            S^{-1}M_1 \otimes_A S^{-1}M_2 \cong
            (M \otimes_A S^{-1}A) \otimes_{S^{-1}A} S^{-1}M_2 \cong
            S^{-1}M_1 \otimes_{S^{-1} A} S^{-1}M_2.
        \end{equation*}
\end{enumerate}

\exercise{2}
Let $A$ be a ring. We define the \textit{support} of an $A$-module $M$
as $\supp(M) \coloneqq \{ \fp \in \spec(A) \mid M_\fp \neq 0\}$.
\begin{enumerate}
    \item Assume $M$ is finitely generated. Show that $\supp(M) 
        = \{\fp \in \spec(A) \mid M \otimes_A k(\fp) \neq 0\}$, where
        $k(\fp) = \mathrm{Quot}(A/\fp)$. 
    \item Assume $M,N$ are finitely generated $A$-modules. Show 
        $\supp(M\otimes_A N) = \supp(M) \cap \supp(N)$.
\end{enumerate}
\textbf{Solution.} 
\begin{enumerate}
    \item We will show that $M_\fp \neq 0$ if and only if $M \otimes_A k(\fp)
        \neq 0$. The map $A \to k(\fp)$ factors through the map 
        $A_\fp \to k(\fp)$, and we find $M \otimes_A k(\fp) = M_\fp
        \otimes_{A_\fp} k(\fp)$, this directly gives the 
        implication $M \otimes_A k(\fp) \neq 0 \implies M_\fp \neq 0$. 

        For the other direction, we use Nakayama's Lemma. It (or at least one version
        of it) states that if $N \neq 0$ is a finitely generated module over a local ring
        $B$ with maximal ideal $I$, we have $IN \neq N$. In our situation, if
        we assume $M_\fp \neq 0$, Nakayama says
        \begin{equation*}
            M_\fp \otimes_{A_\fp} k(\fp) \cong M_{\fp}/\fp M_\fp \neq 0.
        \end{equation*}
        Done.
    \item We'll show that $(M \otimes_A N) \otimes k(\fp) \neq 0$ if and only if
        $M \otimes_A k(\fp) \neq 0$ and $N \otimes_A k(\fp) \neq 0$. 
        Exercise 1.2 gives the isomorphism
        \begin{equation*}
            (M \otimes_A k(\fp)) \otimes_{k(\fp)} (N \otimes_A k(\fp))
            \cong M \otimes_A (N \otimes_{A} k(\fp)) \cong (M \otimes_A N) \otimes_A k(\fp).
        \end{equation*}
        From here we can directly check the desired equivalence.
\end{enumerate}

\exercise{3}
Let $A$ be a ring, let $S \subset A$ be a multiplicative subset and let 
$M,N$ be $A$-modules.
\begin{enumerate}
    \item Assume that $M$ is finitely presented $A$-module. Show that the
        map 
        \begin{equation*}
            S^{-1} \Hom_A(M,N) \to \Hom_{S^{-1}A}(S^{-1}M, S^{-1}N), \quad
            \phi/s \mapsto (m/t \mapsto \phi(m)/st)
        \end{equation*}
        is a well-defined isomorphism.
    \item Construct a counterexample to the above if $M$ is only assumed
        to be finitely generated.
\end{enumerate}
\textbf{Solution.}
\begin{enumerate}
    \item First, note that we always (without hypothesis on $M$) obtain such a
        map. This follows (for example) from exercise 1.1 with $B = S^{-1}A$.
        We obtain the isomorphism
        \begin{equation*}
            \Hom_A(M, S^{-1}N) \xto \sim \Hom_{S^{-1}A}(S^{-1} M, S^{-1}N).
        \end{equation*}
        Also, the natural map $N \to S^{-1}N$ yields a map 
        \begin{equation*}
            \Hom_A(M, N) \to \Hom_A(M,S^{-1}N).
        \end{equation*}
        Finally, as multiplication with any $s \in S$ gives an isomorphism on the right hand
        side, we obtain a morphism
        \begin{equation*}
            S^{-1}\Hom_A(M, N) \to \Hom_A(M,S^{-1}N) \xto \sim \Hom_{S^{-1}A}(S^{-1} M, S^{-1}N).
        \end{equation*}
        by the universal property of localization on modules. One readily checks that 
        this morphism is the one provided by the exercise.

        Now we have to show that this is an isomorphism if $M$ is finitely presented.
        As usual, we write $M$ as part of a short exact sequence
        \begin{equation*}
            0 \to A^m \to A^n \to M \to 0.
        \end{equation*}
    Now we use that $\Hom_A(-,N)$ is right-exact. Hence applying $\Hom_A(-,N)$ yields
    an exact sequence
    \begin{equation*}
        0 \to 0 \to \Hom_A(M, N) \to \Hom_A(A^n,N) \cong N^n \to \Hom_A(A^m,N) \cong A^m.
    \end{equation*}
    Localizing at $S$ is exact, so we obtain
    \begin{equation*}
        0 \to 0 \to S^{-1}\Hom_A(M, N) \to (S^{-1}N)^n \to (S^{-1}N)^m.
    \end{equation*}   
    Similarly, we can localize at $S$ first and then apply
    $\Hom_{S^{-1}A}(S^{-1}(-), S^{-1}N)$, which yields the exact sequence
    \begin{equation*}
        0 \to 0 \to \Hom_{S^{-1}A}(S^{-1}M, S^{-1}N) \to (S^{-1}N)^n \to
        (S^{-1}N)^m.
    \end{equation*}
    Now we can use the $5$-lemma again!
\[\begin{tikzcd}[ampersand replacement=\&]
	0 \& 0 \& {S^{-1}\Hom_A(M,N)} \& {(S^{-1}A)^n} \& {(S^{-1}A)^m} \\
	0 \& 0 \& {\Hom_{S^{-1}A}(S^{-1}M, S^{-1}N)} \& {(S^{-1}A)^n} \& {(S^{-1}A)^m}
	\arrow[Rightarrow, no head, from=1-1, to=2-1]
	\arrow[Rightarrow, no head, from=1-2, to=2-2]
	\arrow[Rightarrow, no head, from=1-4, to=2-4]
	\arrow[Rightarrow, no head, from=1-5, to=2-5]
	\arrow["{\therefore \text{iso}}", from=1-3, to=2-3]
	\arrow[from=1-2, to=1-3]
	\arrow[from=1-1, to=1-2]
	\arrow[from=2-1, to=2-2]
	\arrow[from=2-2, to=2-3]
	\arrow[from=2-3, to=2-4]
	\arrow[from=2-4, to=2-5]
	\arrow[from=1-3, to=1-4]
	\arrow[from=1-4, to=1-5]
\end{tikzcd}\]

\item A standard example seems to be the following. Let $A = k[x, y_1, y_2, \dots]$ be 
    the polynomial ring in variables indexed by $\N$. Let $M = A / (y_1, y_2, \dots)$,
    $N = A/(xy_1, x^2y_2, \dots)$ and $S = \{1,x,x^2,\dots\}$. Now let's compare both 
    sides of the morphism. Note that $M$ is generated by $1$, so that any
    $A$-linear morphism $\phi: M \to N$ is uniquely determined by the value of
    $\phi(1) \in N$. Now we have $0 = y_1 \phi(1) = y_2 \phi(1) = \dots$, which shows
    that any lift $\tilde{\phi(1)} \in R$ is infintely divisible by $x$, hence 
    $\phi(1) = 0$. On the left hand side, we find that
    $S^{-1} M \cong S^{-1}N \cong k[x^{\pm 1}]$, so there are many $S^{-1}A$-linear
    morphisms $S^{-1} M \to S^{-1}N$. 
\end{enumerate}

\exercise{4}
Let $A$ be a principal domain and let $f \in A \setminus \{0\}$ be a 
non-unit. Show that the $A[T]$-module $(f,T) \subset A[T]$ is not flat. 

\textbf{Solution.}
Consider the map given by multiplication with $f$, which we will 
denote as $\phi: A \to A$. It is injective. Note that $A \cong A[T]/(T)$. 
We want to show that $(f,T) \otimes_{A[T]} A$ is not injective, 
showing that $(f,T)$ is not flat. We have an isomorphism (of $A[T]$-modules)
$$(f,T) \otimes_{A[T]} A \cong (f,T) / T (f,T),$$
and $(f,T) \otimes \phi$ corresponds to the endomorphism given by multiplication
with $f$ under this identification.
Now, $T \neq 0$ in $(f,T)/T(f,T)$, but $fT = \phi(T) = 0$. 

\contactend

\end{document}
