\documentclass[a4paper,11pt]{article}
\pagenumbering{arabic}
\usepackage{../environment}

\begin{document}

\begin{center}
    \huge{Solutions to Sheet 2}
\end{center}

\exercise{1}
Define $\zeta = \frac{-1 + \sqrt{-3}}2 \in \C$. 
\begin{enumerate}
    \item Show that $\zeta$ is a primitive third root of unity.
    \item Show that the norm (for the field extension $\Q(\zeta)/\Q$ of
        an element $x + y\zeta \in \Q(\zeta)$, where $x,y \in \Q$, is given by
        $x^2 - xy + y^2$, and that this is non-negative for all $x,y \in \Q$. 
    \item  Following the discussion of $\Z[i]$ from the lecture, show that a prime
        $p \neq 3$ is of the form $p = x^2 - xy + y^2$ for some $x,y \in \Z$ if
        and only if $p \equiv 1$ (mod $3$). 
\end{enumerate}

\textbf{Solution.}
\begin{enumerate}
    \item We have 
        $$\zeta^3 = \left(\frac 12 (-1 + \sqrt{-3})\right)^3 = 1/8(-1 +
        3\sqrt{-3} - 9 + 3 \sqrt{-3}) = 1.$$ 
        As $\zeta \neq 1$ (and $3$ has no non-trivial
        divisors), it is a primitive (third) root. 
    \item The norm is defined as the product of all galois-conjugates. The minimal
        polynomial of $\zeta$ is given by $f(x) = x^2 + x + 1 = (x - \zeta)(x -
        \bar \zeta)$, so the only non-trivial element in the Galois-group
        $\Gal(\Q(\zeta)/\Q)$ is given by the action defined via 
        $\zeta \mapsto \bar \zeta$, which is the same as complex conjugation. 
        We find 
        $$\Norm(x+\zeta y) = (x+\zeta y)(x + \bar \zeta y) = x^2 + (\zeta +
        \bar \zeta)xy + \zeta \bar \zeta y^2.$$
        The claim follows as $\zeta \bar \zeta$ and $\zeta + \bar \zeta$
        are given by the constant and the negative of the second-to-highest
        coefficient of the minimal polynomial of $\zeta$ (which are both given by
        $1$).


\end{enumerate}

\exercise{2}
\begin{enumerate}
    \item Let $A$ be a principal ideal domain that is not a field, and let $\fm
        \subset A$ be a maximal ideal. Prove that $\fm^n/\fm^{n+1}$ is a
        one-dimensional vector space over $A/\fm$ for any $n \geq 0$. 
    \item Let $A = \C[x,y]$ and $\fm = (x,y)$. Compute $\dim_{A/\fm} (\fm^n
        /\fm^{n+1})$ for $n \geq 0$. Deduce that $A$ is not a principal ideal
        domain. 
    \item Let $A = \Z[\sqrt{-3}]$. Show that $A$ has a unique maximal ideal $\fm$
        with $\fm \cap \Z = (2)$. Compute $\dim_{A/\fm} \fm/\fm^2$. Deduce that 
        $A$ is not a principal ideal domain.
\end{enumerate}

\textbf{Solution.}
\begin{enumerate}
    \item Let $\pi \in A$ such that $(\pi) = \fm$. We have the map (of $A$-modules)
        $$\phi: A \to \fm^n/\fm^{n+1}, \quad a \mapsto a \pi^n$$. 
        It is obviously surjective, and one quickly verifies that the kernel is
        given by $(\pi)$. Hence we find $A/\fm \cong \fm^n/\fm^{n+1}$, and we
        are done. 
    \item We have $\fm^n = (x^n, x^{n-1}y, \dots, xy^{n-1},y^n)$. These generators
        form a basis for $\fm^{n}/\fm^{n+1}$ (they are generating and linearly
        independent over $\C$), hence the dimension is $n+1$. This contradicts
        what we showd for principal ideal domains once $n \geq 1$. 
\end{enumerate}

\exercise{3}
Let $A$ be a unique factorization domain. 
\begin{enumerate}
    \item Show that for any prime element $\pi \in A$, the ideal $\fp = (\pi)$ is 
        prime and only contains the prime ideals $\{0\}$ and $\fp$.
    \item Conversely, let $0 \neq \fp \subset A$ be a prime ideal such that $\{0\}$
        and $\fp$ are the only prime ideals of $A$ that are contained in $\fp$. 
        Show that $\fp = (\pi)$ for some prime element $\pi \in A$. 
    \item Assume that each non-zero prime ideal $\fp \subset A$ satisfies the
        assumption in 2). Show that $A$ is a principal ideal domain. 
\end{enumerate}

\exercise{4}
\begin{enumerate}
    \item Let $A$ be any ring. Show that $A$ contains minimal prime ideals. 
    \item Determine the minimal prime ideals of $\Z[x,y]/(xy)$. 
\end{enumerate}
\contactend
\end{document}
