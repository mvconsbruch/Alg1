\documentclass[a4paper,11pt]{article}
\pagenumbering{arabic}
\usepackage{../environment}

\begin{document}

\begin{center}
    \huge{Solutions to Sheet 2}
\end{center}

\exercise{1}
Define $\zeta = \frac{-1 + \sqrt{-3}}2 \in \C$. 
\begin{enumerate}
    \item Show that $\zeta$ is a primitive third root of unity.
    \item Show that the norm (for the field extension $\Q(\zeta)/\Q$ of
        an element $x + y\zeta \in \Q(\zeta)$, where $x,y \in \Q$, is given by
        $x^2 - xy + y^2$, and that this is non-negative for all $x,y \in \Q$. 
    \item  Following the discussion of $\Z[i]$ from the lecture, show that a prime
        $p \neq 3$ is of the form $p = x^2 - xy + y^2$ for some $x,y \in \Z$ if
        and only if $p \equiv 1$ (mod $3$). 
\end{enumerate}

\textbf{Solution.}
\begin{enumerate}
    \item We have 
        $$\zeta^3 = \left(\frac 12 (-1 + \sqrt{-3})\right)^3 = 1/8(-1 +
        3\sqrt{-3} - 9 + 3 \sqrt{-3}) = 1.$$ 
        As $\zeta \neq 1$ (and $3$ has no non-trivial
        divisors), it is a primitive (third) root. 
    \item The norm is defined as the product of all galois-conjugates. The minimal
        polynomial of $\zeta$ is given by $f(x) = x^2 + x + 1 = (x - \zeta)(x -
        \bar \zeta)$, so the only non-trivial element in the Galois-group
        $\Gal(\Q(\zeta)/\Q)$ is given by the action defined via 
        $\zeta \mapsto \bar \zeta$, which is the same as complex conjugation. 
        We find 
        $$\Norm(x+\zeta y) = (x+\zeta y)(x + \bar \zeta y) = x^2 + (\zeta +
        \bar \zeta)xy + \zeta \bar \zeta y^2.$$
        The claim follows as $\zeta + \bar \zeta = -1$ and $\zeta \bar \zeta = 1$. 

        It remains to show that the norm is always positive. The claim is 
        trivial if $x,y$ have different sign. If the sign is the same, we may
        wlog assume that both are positive. In that case, this is a special case
        of the AM-GM inequality. But for completeness, here is a calculation:
        $$ x^2 -xy +y^2 \geq x^2 - 2xy + y^2 = (x-y)^2 \geq 0$$

    \item We want to show that there is an element $z = x + \zeta y \in \Z(\zeta)$ 
        with $\Norm(z) = p$ if and only if $3 \mid p-1$. We know from the lecture
        that $\Z[\zeta]$ is a principal ideal domain. 
        First note that the "only if" part is trivial. Indeed, we have
        \begin{equation*}
            x^2 -xy + y^2 \equiv \begin{cases}
                1 \pmod 3, &\text{ if } (x,y) = (1,1), (0,1), (1,0)\\
                0 \pmod 3, &\text{ if } (x,y) = (0,0).
            \end{cases}
        \end{equation*}
        If $3 \mid x$ and $3 \mid y$ we find that $3 \mid \Norm(x+\zeta y)$,
        hence $\Norm(x+\zeta y)$ cannot be a prime. This shows that all primes
        of the form $x^2 -xy + y^2$ have residue $1$ mod $3$. 

        To show the converse implication, let $p \in \Z$ be any prime. 
        As $\Z[\zeta]$ is a PID, the prime elements $\pi \in \Z[\zeta]$
        that divide $p$ are in bijection with the maximal (equivalently,
        non-zero prime) ideals $\fm \subset \Z[\zeta]$ such that $\fm \cap \Z =
        (p)$. An easy computation shows (lecture 3) that these ideals are in
        bijection with the irreducible monic factors of $T^2 + T + 1$ in
        $\FF_p[T]$. As $\FF_p[T]$ has a non-trivial third root of unity if and
        only if $3 \mid p-1$, we find that there are 
        two prime ideals "above" $(p)$ if $3 \mid p-1$.

        Hence, let $\pi_1, \pi_2$ be the two prime elements of 
        $\Z[\zeta]$ that divide $p$ and write $(p) = (\pi_1^{e_1})(\pi_2^{e_2})$. 
        As in the lecture we find $\Norm(\pi_1) = \Norm(\pi_2) = p$, which 
        implies $e_1 = e_2 = 1$. Now we have a primary decomposition 
        $p = \pi_1 \pi_2$, which implies that $\pi_1 = \bar \pi_2$, which
        gives the desired representation of $p$. 

\end{enumerate}

\exercise{2}
\begin{enumerate}
    \item Let $A$ be a principal ideal domain that is not a field, and let $\fm
        \subset A$ be a maximal ideal. Prove that $\fm^n/\fm^{n+1}$ is a
        one-dimensional vector space over $A/\fm$ for any $n \geq 0$. 
    \item Let $A = \C[x,y]$ and $\fm = (x,y)$. Compute $\dim_{A/\fm} (\fm^n
        /\fm^{n+1})$ for $n \geq 0$. Deduce that $A$ is not a principal ideal
        domain. 
    \item Let $A = \Z[\sqrt{-3}]$. Show that $A$ has a unique maximal ideal $\fm$
        with $\fm \cap \Z = (2)$. Compute $\dim_{A/\fm} \fm/\fm^2$. Deduce that 
        $A$ is not a principal ideal domain.
\end{enumerate}

\textbf{Solution.}
\begin{enumerate}
    \item Let $\pi \in A$ such that $(\pi) = \fm$. We have the map (of $A$-modules)
        $$\phi: A \to \fm^n/\fm^{n+1}, \quad a \mapsto a \pi^n.$$
        It is obviously surjective, and one quickly verifies that the kernel is
        given by $(\pi)$. Hence we find $A/\fm \cong \fm^n/\fm^{n+1}$, and we
        are done. 
    \item We have $\fm^n = (x^n, x^{n-1}y, \dots, xy^{n-1},y^n)$. These generators
        form a basis for $\fm^{n}/\fm^{n+1}$ (they are generating and linearly
        independent over $\C$), hence the dimension is $n+1$. This contradicts
        what we showd for principal ideal domains once $n \geq 1$. 
    \item We first show that there is a unique maximal ideal of $A$ with 
        $\Z \cap \fm = (2)$. Indeed, those maximal ideals are in bijection with
        the maximal ideals of $\FF_2[T]/(T^2 + 3)$. As $T^2 + 3$ factors in 
        $\FF_2[T]$ as $(T+1)^2$, we find that $\fm = (2, \sqrt{-3} + 1)$ is the
        unique maximal ideal of $\Z[\sqrt{-3}]$ above $(2)$. 

        Now $\fm^2 = (4, 2 \sqrt{-3} + 2, -2 + 2\sqrt{-3})$. Hence the elements
        $2$ and $\sqrt{-3}+1$ do not lie in $\fm^2$ as they have norm $4$ (after
        choosing an embedding into $\C$), while all elements generating
        $\fm^2$ have norm $16$. Hence there are at least $3$ elements in 
        $\fm/\fm^2$, thereby $\dim_{\FF_2}{\fm/\fm^2} \neq 1$. 
\end{enumerate}

\exercise{3}
Let $A$ be a unique factorization domain. 
\begin{enumerate}
    \item Show that for any prime element $\pi \in A$, the ideal $\fp = (\pi)$ is 
        prime and only contains the prime ideals $\{0\}$ and $\fp$.
    \item Conversely, let $0 \neq \fp \subset A$ be a prime ideal such that $\{0\}$
        and $\fp$ are the only prime ideals of $A$ that are contained in $\fp$. 
        Show that $\fp = (\pi)$ for some prime element $\pi \in A$. 
    \item Assume that each non-zero prime ideal $\fp \subset A$ satisfies the
        assumption in 2). Show that $A$ is a principal ideal domain. 
\end{enumerate}

\textbf{Solution.}
\begin{enumerate}
    \item Let $0 \neq \fq$ be a prime contained in $\fp$. Take some nonzero
        element $q \in \fq$. Write $q = a \pi^n$, where $a \in A$ is an element
        not divisible by $\pi$. Now, as $\fq$ is prime, either $\pi^n \in \fq$ or 
        $a \in \fq$. But we have $a \not \in (\pi) \subset \fq$, hence
        $\pi^n \in \fq$. Induction shows that $\pi \in \fq$, which results in
        $\fq = \fp$. 

    \item Suppose $\pi \in \fp$ is a prime element contained in $\fp$. Then
        $(\pi) \subset \fp$, which by assumption shows $(\pi) = \fp$. 
        We only need to show that there are prime ideals in any
        nonzero element $\fp$. For that sake, let $a \in \fp$. There is a finite
        decomposition $a = \prod_{i=1}^n p_i^{e_i}$, and we find 
        that for some $i$, the prime element $p_i$ lies in $\fp$. 

    \item Let $I \neq (0)$ be any ideal. Let $\pi_1, \dots, \pi_n$ be the
        finite set of primes such that $I \subset (\pi_i)$ (this is a finite set
        because any $f \in I$ has only a finite number of divisors), and let
        $e_i$ be the maximal integer such that $I \subset (\pi_i^{e_i})$ holds.
        Write $\alpha = \pi_1^{e_1} \dots \pi_n^{e_n}$. We claim that $I =
        (\alpha)$. The inclusion "$I \subset (\alpha)$" is trivial. 

        To show the other direction, it suffices to show that $\alpha \in I$. 
        Suppose that $I = (g_i \mid i \in I)$. Write $g_i = h_i \alpha$ and inspect
        the ideal $I' = (h_i \mid i \in I)$. By construction there is no 
        prime $\pi \in A$ such that $I' \subset (\pi)$, otherwise the
        factors $e_i$ would not have been chosen maximal. But this shows that 
        $I' = (1)$, i.e., $\alpha \in I$. 


\end{enumerate}

\exercise{4}
\begin{enumerate}
    \item Let $A$ be any ring. Show that $A$ contains minimal prime ideals. 
    \item Determine the minimal prime ideals of $\Z[x,y]/(xy)$. 
\end{enumerate}

\textbf{Solution.}
\begin{enumerate}
    \item What does Zorn's Lemma say again? Ah. If in an ordered set we can
        show that any totally ordered chain has a minimal element, then there
        are minimal elements. As our ordered set we take the set of prime ideals,
        ordered by inclusion. To apply Zorn's lemma, let $\fp_1 \supset \fp_2
        \supset \dots$ be a decreasing chain of prime ideals. We need to show
        that this chain has a minimal element, which is a prime ideal 
        $\fp$ such that $\fp_i \supset \fp$. We set $\fp = \bigcap_{i \in \N}
        \fp_i$, and we have to show that this is a prime ideal. This is
        straight-forward. Assume that $ab \in \fp$. Assume $b \not \in \fp$.
        Then, there is some $i$ such that $b \not \in \fp_i$, and hence $b \not
        \in \fp_j$ for all $j \geq i$. But now, as all of the $\fp_i$ are
        prime, we find that $a \in \fp_i$ for all $i$. Hence $a \in \fp$, and
        we are done. 
    \item We use that minimal prime ideals of $\Z[x,y]/(xy)$ are exactly those
        prime ideals of $\Z[x,y]$ that are minimal among those containing
        $(xy)$. Using that $\Z[x,y]$ is a UFD, we find that those prime ideals
        are given by $(x)$ and $(y)$.
\end{enumerate}

\contactend
\end{document}
