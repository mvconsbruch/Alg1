\documentclass[a4paper,11pt]{article}
\pagenumbering{arabic}
\usepackage{../environment}

\begin{document}

\begin{center}
    \huge{Solutions to Sheet 2}
\end{center}

\exercise{1}
Define $\zeta = \frac{-1 + \sqrt{-3}}2 \in \C$. 
\begin{enumerate}
    \item Show that $\zeta$ is a primitive third root of unity.
    \item Show that the norm (for the field extension $\Q(\zeta)/\Q$ of
        an element $x + y\zeta \in \Q(\zeta)$, where $x,y \in \Q$, is given by
        $x^2 - xy + y^2$, and that this is non-negative for all $x,y \in \Q$. 
    \item  Following the discussion of $\Z[i]$ from the lecture, show that a prime
        $p \neq 3$ is of the form $p = x^2 - xy + y^2$ for some $x,y \in \Z$ if
        and only if $p \equiv 1$ (mod $3$). 
\end{enumerate}

\textbf{Solution.}
\begin{enumerate}
    \item We have 
        $$\zeta^3 = \left(\frac 12 (-1 + \sqrt{-3})\right)^3 = 1/8(-1 +
        3\sqrt{-3} - 9 + 3 \sqrt{-3}) = 1.$$ 
        As $\zeta \neq 1$ (and $3$ has no non-trivial
        divisors), it is a primitive (third) root. 
    \item The norm is defined as the product of all galois-conjugates. The minimal
        polynomial of $\zeta$ is given by $f(x) = x^2 + x + 1 = (x - \zeta)(x -
        \bar \zeta)$, so the only non-trivial element in the Galois-group
        $\Gal(\Q(\zeta)/\Q)$ is given by the action defined via 
        $\zeta \mapsto \bar \zeta$, which is the same as complex conjugation. 
        We find 
        $$\Norm(x+\zeta y) = (x+\zeta y)(x + \bar \zeta y) = x^2 + (\zeta +
        \bar \zeta)xy + \zeta \bar \zeta y^2.$$
        The claim follows as $\zeta \bar \zeta$ and $\zeta + \bar \zeta$
        are given by the constant and the negative of the second-to-highest
        coefficient of the minimal polynomial of $\zeta$ (which are both given by
        $1$).


\end{enumerate}

\exercise{2}
\begin{enumerate}
    \item Let $A$ be a principal ideal domain that is not a field, and let $\fm
        \subset A$ be a maximal ideal. Prove that $\fm^n/\fm^{n+1}$ is a
        one-dimensional vector space over $A/\fm$ for any $n \geq 0$. 
    \item Let $A = \C[x,y]$ and $\fm = (x,y)$. Compute $\dim_{A/\fm} (\fm^n
        /\fm^{n+1})$ for $n \geq 0$. Deduce that $A$ is not a principal ideal
        domain. 
    \item Let $A = \Z[\sqrt{-3}]$. Show that $A$ has a unique maximal ideal $\fm$
        with $\fm \cap \Z = (2)$. Compute $\dim_{A/\fm} \fm/\fm^2$. Deduce that 
        $A$ is not a principal ideal domain.
\end{enumerate}

\textbf{Solution.}
\begin{enumerate}
    \item Let $\pi \in A$ such that $(\pi) = \fm$. We have the map (of $A$-modules)
        $$\phi: A \to \fm^n/\fm^{n+1}, \quad a \mapsto a \pi^n$$. 
        It is obviously surjective, and one quickly verifies that the kernel is
        given by $(\pi)$. Hence we find $A/\fm \cong \fm^n/\fm^{n+1}$, and we
        are done. 
    \item We have $\fm^n = (x^n, x^{n-1}y, \dots, xy^{n-1},y^n)$. These generators
        form a basis for $\fm^{n}/\fm^{n+1}$ (they are generating and linearly
        independent over $\C$), hence the dimension is $n+1$. This contradicts
        what we showd for principal ideal domains once $n \geq 1$. 
    \item 
\end{enumerate}

\exercise{3}
Let $A$ be a unique factorization domain. 
\begin{enumerate}
    \item Show that for any prime element $\pi \in A$, the ideal $\fp = (\pi)$ is 
        prime and only contains the prime ideals $\{0\}$ and $\fp$.
    \item Conversely, let $0 \neq \fp \subset A$ be a prime ideal such that $\{0\}$
        and $\fp$ are the only prime ideals of $A$ that are contained in $\fp$. 
        Show that $\fp = (\pi)$ for some prime element $\pi \in A$. 
    \item Assume that each non-zero prime ideal $\fp \subset A$ satisfies the
        assumption in 2). Show that $A$ is a principal ideal domain. 
\end{enumerate}

\textbf{Solution.}
\begin{enumerate}
    \item Let $0 \neq \fq$ be a prime contained in $\fp$. Take some nonzero
        element $q \in \fq$. Write $q = a \pi^n$, where $a \in A$ is an element
        not divisible by $\pi$. Now, as $\fq$ is prime, either $\pi^n \in \fq$ or 
        $a \in \fq$. But we have $a \not \in (\pi) \subset \fq$, hence
        $\pi^n \in \fq$. Induction shows that $\pi \in \fq$, which results in
        $\fq = \fp$. 

    \item Suppose $\pi \in \fp$ is a prime element contained in $\fp$. Then
        $(\pi) \subset \fp$, which by assumption shows $(\pi) = \fp$. 
        We only need to show that there are prime ideals in any
        nonzero element $\fp$. For that sake, let $a \in \fp$. There is a finite
        decomposition $a = \prod_{i=1}^n p_i^{e_i}$, and we find 
        that for some $i$, the prime element $p_i$ lies in $\fp$. 

    \item Let $I \neq (0)$ be any ideal. Let $\pi_1, \dots, \pi_n$ be the
        finite set of primes such that $I \subset (\pi_i)$ (this is a finite set
        because any $f \in I$ has only a finite number of divisors), and let
        $e_i$ be the maximal integer such that $I \subset (\pi_i^{e_i})$ holds.
        We claim that $I = (\pi_1^{e_1} \cdots \pi_n^{e_n})$. We write $(f)$
        for the right hand side. The inclusion "$I \subset (f)$" is trivial. 

        To show the other direction, it suffices to show that $f \in I$. By 
        construction, there is some $g_i \in I$ such that $\pi_i^{e_i + 1}
        \nmid g_i$. 


\end{enumerate}

\exercise{4}
\begin{enumerate}
    \item Let $A$ be any ring. Show that $A$ contains minimal prime ideals. 
    \item Determine the minimal prime ideals of $\Z[x,y]/(xy)$. 
\end{enumerate}

\textbf{Solution.}
\begin{enumerate}
    \item What does Zorn's Lemma say again? Ah. If in an ordered set we can
        show that any totally ordered chain has a minimal element, then there
        are minimal elements. As our ordered set we take the set of prime ideals,
        ordered by inclusion. To apply Zorn's lemma, let $\fp_1 \supset \fp_2
        \supset \dots$ be a decreasing chain of prime ideals. We need to show
        that this chain has a minimal element, which is a prime ideal 
        $\fp$ such that $\fp_i \supset \fp$. We set $\fp = \bigcup_{i \in \N}
        \fp_i$, and we have to show that this is a prime ideal. This is
        straight-forward. Assume that $ab \in \fp$. Assume $b \not \in \fp$.
        Then, there is some $i$ such that $b \not \in \fp_i$, and hence $b \not
        \in \fp_j$ for all $j \geq i$. But now, as all of the $\fp_i$ are
        prime, we find that $a \in \fp_i$ for all $i$. Hence $a \in \fp$, and
        we are done. 
    \item We use that minimal prime ideals of $\Z[x,y]/(xy)$ are exactly those
        prime ideals of $\Z[x,y]$ that are minimal among those containing
        $(xy)$. As $\Z[x,y]$ is a UFD, the only prime ideals containing 
        $(xy)$ are $(x)$ and $(y)$.
\end{enumerate}

\contactend
\end{document}
