\documentclass[a4paper,11pt]{article}
\pagenumbering{arabic}
\usepackage{../environment}

\begin{document}

\begin{center}
    \huge{Solutions to Sheet 1}
\end{center}

\exercise{1}
Determine the nilradical, the Jacobson radical and the units for each
ring $A$ below:
\begin{enumerate}
    \item $k$ a field and $A = k[T]$,
    \item $k$ a field and $A = k[\epsilon, T]/(\epsilon^2)$,
    \item $n \geq 1$, $k$ a field and $A = k \llbracket T_1, \dots, T_n \rrbracket$. 
\end{enumerate}
\textbf{Solution.}
\begin{enumerate}
    \item \textit{Nilradical.} If $B$ is any commutative ring without
        zero divisors, then $B[T]$ doesn't have zero divisors. Indeed,
        if $f, g \in B[T]$ with $fg = 0$, we can look at the leading terms
        of $f$ and $g$, obtaining $f = 0$ or $g = 0$. We now obtain $\Nil(A) = (0)$
        as every element in the nilradical is a zero divisor.

        \textit{Units.} Obviously, $k^\times \subset k[T]^\times$.
        We have the additive degree map $\deg: k[T]^\times \to \N_0$. If
        we have elements $f,g \in k[T]$ with $fg = 1$,
        then $0 = \deg (fg) = \deg(f) + \deg(g)$, thereby $\deg(f) = \deg(g) = 0$
        and $f,g \in k^\times$. This shows that $k^\times \supset k[T]^\times$,
        and we have equality. 

        \textit{Jacobson radical.} Note that if $B$ is any commutative 
        ring and $f \in \Jac(B)$, then $1 + f \in B^\times$. Indeed, if we had
        $1 + f \not \in B^\times$, we'd find some maximal ideal $\fm$ containing
        $1 + f$ (by Zorn's lemma). But now $f \in \fm$ (as $f \in \Jac(B)$)
        and $1 + f \in \fm$, hence $1 \in \fm$. This is a contradiction. 
        Thereby we obtain
        that every $f \in \Jac(A)$ has degree $0$, i.e., lies in $k$. As
        $A^\times \cap \Jac(A) = \emptyset$, we find $\Jac(A) = 0$. (As
        $\Jac(A) \supset \Nil(A)$, this is stronger than $\Nil(A) = 0$.) 
    \item \textit{Nilradical and Jacobson radical.} We claim that if $I \subset
        \Nil(A)$, there is an equality $\Nil(A)/I = \Nil(A/I)$. Indeed, this
        can be seen directly by writing the nilradical as the intersection of
        prime ideals. The same statement is true for the Jacobson radical. 

        We apply this statement with $I = (\varepsilon)$. As $\varepsilon^2 = 0$, 
        we have $I \subset \fp$ for every prime ideal, hence $(\varepsilon)
        \subset \Jac(A)$. As $A/(\varepsilon) \cong k[T]$, we have 
        $(0) = \Nil(A/(\varepsilon)) = \Nil(A) / (\varepsilon)$. This shows 
        $\Nil(A) = (\varepsilon)$. 

        The same proof, but with $\Jac$ in place of $\Nil$ (and maximal ideals
        instead of prime ideals) shows that $\Jac(A) = (\varepsilon)$.

        \textit{Units.} There are probably smarter ways to do this, but let's try
        brute force. Suppose we have $f = f_1 + \varepsilon f_2$ and 
        $g = g_1 + \varepsilon g_2$, where $f_i, g_i \in k[T]$, such that $fg =
        1$. Now $1 = f_1 g_1 + \varepsilon (f_1 g_2 + f_2 g_1)$. It follows 
        that $f_1 \in k^\times$, and we clam that this is also sufficient for 
        $f \in A^\times$. Indeed, up to multiplication with a constant 
        in $k^\times$, $f$ is of the form $1 + \varepsilon f_2$, and now
        $f$ admits an inverse $f^{-1} = 1 - \varepsilon f_2$. 
        
    \item \textit{Units.} We first claim that every $f \in A$ with non-zero 
        constant term is invertible. Indeed, after multiplying with a unit
        $c \in k^\times$ we may assume that $f = 1 + R$ with $R \in (T_1,
        \dots, T_n)$. Now, $f$ admits the inverse $f^{-1} = \frac 1{1-(1-f)} =
        \sum_{n=0}^\infty (1-f)^n \in k\llbracket T_1, \dots, T_n \rrbracket.$

        \textit{Jacobson radical.} We first claim that $A$ is a local ring, i.e., a 
        ring with a unique maximal ideal. Indeed, we have seen that every
        element not lying in the ideal $\fm = (T_1, \dots, T_n)$ is invertible, hence
        $\fm$ is an ideal that contains all other ideals.

        \textit{Nil radical.} We want to show that $A$ is reduced. 
        More generally, we prove the following statement, from where the claim
        follows by induction.
        \begin{equation*}
            \textit{If $B$ is reduced, $B\llbracket T \rrbracket$ is reduced.}
        \end{equation*}
        for the sake of contradiction, assume that $f \in B\llbracket T
        \rrbracket$ is a non-zero power series with $f^n = 0$. Write $f = a_d
        T^d + a_{d+1}T^{d+1} + \dots$ with $a_d \neq 0$. Now $f^n = 0$ implies
        $a_d^n = 0$, so $a_d = 0$ by reducedness of $B$. Hence $f = 0$. 
\end{enumerate}
    
\exercise{2}
Prove the \textit{Chinese remainder theorem}: Let $A$ be a ring and $\fa, \fb
\subset A$ two ideals such that $\fa + \fb = A$. Then the map 
\begin{equation*}
    A/ \fa \cap \fb \to A/ \fa \times A/\fb, \quad 
    r + \fa \cap \fb \mapsto (r + \fa, r + \fb)
\end{equation*}
is an isomorphism. Moreover, show that $\fa \cap \fb = \fa \cdot \fb$, where
$\fa \cdot \fb$ is the smalles ideal in $A$ containing all products $ab$ 
wth $a \in A$, $b \in B$. 
Show $a \cap b = ab$. Show that map has kernel $a \cap b$ and that homomorphism is surjective. 

\textbf{Solution.} We first show that this map is well-defined, and indeed a 
homomorphism of rings. This is evident for the reduction-mod-$\fa$ and 
reduction-mod-$\fb$ maps $A \to A/\fa$ and $A/\fb$. By the universal property of 
the product of rings we obtain the map $A \to A/\fa \times A/\fb$. The kernel
of this homomorphism is given by the elements in $A$ which lie simultaneously in 
$\fa$ and $\fb$, hence we obtain an injective map 
\begin{equation*}
    A/(\fa \cap \fb) \to A/\fa \times A/\fb.
\end{equation*}
To show surjectivity, it suffices to construct elements $a, b \in A$ such that 
$a \mapsto (0,1)$ and $a \mapsto (1,0)$. As $\fa + \fb = A$, there are elements
$a \in \fa$ and $b \in \fb$ such that $a + b = 1$. These are the elements we 
are looking for! Indeed, as $a = 1 - b$ we find that $a$ reduces to $1$ mod $\fb$,
and as $a \in \fa$ we find $(a + \fa, a + \fb) = (\fa, 1 + \fb)$. 

\textbf{Remark.} There is a more general version of the chinese remainder theorem
which we will need in exercise 4. Namely, if $\fa_1, \dots, \fa_n$ is a finite
set of pairwise coprime ideals (meaning that for any choice $1 \leq i < j \leq n$
we have $\fa_i + \fa_j = A$), there is an isomorphism
\begin{equation*}
    A/(\fa_1 \cap \dots \cap \fa_n) \cong A/\fa_1 \times \dots \times A/\fa_n.
\end{equation*}
To see this, one can either generalize the proof given above, or use induction
after showing that the coprimality assumption implies that the ideals
$(\fa_1 \cap \dots \cap \fa_{n-1})$ and $\fa_n$ are coprime.

We now show that $\fa \cap \fb = \fa \cdot \fb$. The inclusion 
$\fa \cap \fb \supset \fa \cdot \fb$ is obvious, as all products $ab$ lie in both
$\fa$ and $\fb$. To show the reverse inclusion, let $f \in \fa \cap \fb$. Again,
let $a + b = 1$ with $a \in \fa$ and $b \in \fb$. Then $fa + fb = f$, and the 
left hand side lies in $\fa \cdot \fb$ by definition. 

\textbf{Remark.} Note that this statement
is wrong if we drop the assumption that $\fa + \fb = 1$. Indeed, take for 
example $\fa = (4)$, $\fb = (6)$ as ideals of $\Z$. Then $\fa \fb = (24)$, 
while $\fa \cap \fb = (12)$. However, the assumption that $\fa + \fb = A$ is 
not necessary. In the case $A = k[X,Y]$, $\fa = (X)$ and $\fb = (Y)$ we still 
have $\fa \fb = (XY) = \fa \cap \fb$ even though $\fa + \fb = (X,Y) \neq A$.


\exercise{3}
Recall that an element $e \in A$ in a ring $A$ is called idempotent if $e^2 = e$. 
\begin{enumerate}
    \item Let $A$ be a ring. Show that the map $e \mapsto (A_1 \coloneqq eA,
        A_2 \coloneqq (1-e)A)$ induces a bijection between the set $\Idem(A)$
        of idempotents of $A$
        and the set of decompositions $A = A_1 \times A_2$ of rings. 
    \item Let $A = \Z/133\Z$. Determine $\Idem(A)$. 
\end{enumerate}

\textbf{Solution.}
\begin{enumerate}
    \item The exercise does not make clear what it means by a decomposition. In
        the scope of this exercise, a decomposition of $A$ is an
        isomorphism $\delta : A \to A_1 \times A_2$, where 
        $A_1$ and $A_2$ are any two rings. We say that two decompositions
        $\delta_1: A \to A_1 \times A_2$ and $\delta_2: A \to B_1 \times B_2$
        are isomorphic iff there are isomorphisms
        $\phi_i : A_i \to B_i$, $i = 1,2$ such that $(\phi_1, \phi_2) \circ
        \delta_1 = \delta_2$. We define the set $D_A$ as the set of isomorphism
        classes of the set\footnote{Actually I'm not sure if this really is a
        set, but whatever.  The decompositions will certainly form a category
        (a groupoid), with morphisms the isomorphisms we described. The
        isomorphism classes do form a set as they all are represented by quotients
        of $A$.} of decompositions, and we'll show that the map specified in the
        exercise gives a bijection $\Idem(A) \to D_A$. 

        First, note that $(1-e)^2 = (1-e)$ for any idempotent $e$. 

        We have show that the map really is a map! That is, we show
        that for any idempotent element $e \in A$, there is an isomorphism
        $\delta_e: A \cong eA \times (1-e)A$, where $eA$ and $(1-e)A$ carry the
        ring structure of $A$, but with identity given by $e$ and $(1-e)$,
        respectively. Surjectivity is clear, and injectivity boils down to the 
        calculation $\ker(\delta_e) = (e) \cap (1-e) = (e)\cdot(1-e) = (0)$. 

        Next, note that we also have a map $D_A \to \Idem(A)$ given by sending
        $\delta: A \to A_1 \times A_2$ to $e_\delta \coloneqq \delta^{-1}(1,0)$. 
        This map does not depend on the isomorphism class of $\delta$ as ring
        homomorphisms preserve the multiplicative unit. One quickly verifies that
        $\Idem(A) \to D_A \to \Idem(A)$ is the identity. The last thing to see 
        is that $D_A \to \Idem(A) \to D_A$ is the identity as well, which is the 
        same as showing that for a given decomposition $\delta: A \to A_1
        \times A_2$, there is an isomorphism $\delta \cong \delta_{e_\delta}$. 
        Such an isomorphism is the same as isomorphisms $\phi_1 : e_\delta A
        \to A_1$, $\phi_2: (1-e_\delta)A \to A_2$. As $\delta$ sends the ideal
        $(e) \subset A$ to the ideal generated by $(1,0)$ in $A_1 \times A_2$,
        it is evident that there are such isomorphisms. 
    \item Note that $133 = 19 \times 7$, hence by the chinese remainder theorem
        $\Z/133 \cong \Z/19 \times \Z/7$. The right hand side is a product of 
        fields, and it is clear that the only idempotents there are given by 
        $(0,0), (1,0), (0,1), (1,1)$. As $1= 19 \cdot 3 - 7 \cdot 8$, 
        the isomorphism from the chinese remainder theorem is given by 
        $(a,b) \mapsto 57b + 77a$, and we find that the non-trivial idempotents
        are given by $57$ and $77$. 
\end{enumerate}


\exercise{4}
Let $k$ be a field and let $k \to A$ be a ring homomorphism such that 
$A$ is finite dimensional over $k$ (i.e., regarded as a $k$-vector space,
$A$ has finite dimension).
\begin{enumerate}
    \item Show that $A$ is a field if $A$ is an integral domain.
    \item Deduce that each prime ideal in $A$ is maximal.
    \item Deduce that if $A$ is reduced, then $A$ is isomorphic to a finite 
        product of finite field extensions $l/k$. 
\end{enumerate}
\textbf{Solution.}
\begin{enumerate}
    \item Let $x \in A$ be nonzero. Let $\phi: A \to A$ be the map obtained by 
        multiplication with $x$, i.e., $\phi(a) = x a$. Now $\phi$ is a morphism
        of $k$-vector spaces (as $\phi(\lambda a + b) = \lambda \phi(a) + \phi(b)$
        for $\lambda \in k$, $a,b \in A$.), and it is injective by the fact that 
        $A$ is an integral domain. Indeed, if $xa = 0$, we find $a = 0$ as there
        are no zero divisors and $x \neq 0$. But now $\phi$ is an injective morphism
        between $k$-vector spaces of the same dimension, hence an isomorphism.
        In particular, we find some element $x^{-1} \in A$ such that 
        $1 = \phi(x^{-1}) = x x^{-1}$. Hence every non-zero element of $A$ has an
        inverse, and $A$ is a field.
    \item Let $\fp \in A$ be a prime ideal. We apply what we showed in part 1)
        to $A/\fp$. As $\fp$ is prime, $A/\fp$ is an integral domain. But also,
        the composition $k \to A \to A/ \fp$ turns $A/\fp$ into a $k$-vector space
        with $\dim_k(A/\fp) \leq \dim_k(A)$ (surjective maps between vector spaces
        reduce dimension). In particular, $A/\fp$ is finite-dimensional over
        $k$. Now part 1) gives that $A/\fp$ is a field, and as an ideal is 
        maximal if and only it's quotient ring is a field, we find that 
        $\fp$ is maximal. 
    \item Let $M$ be the set of maximal (or prime, they are the same by the
        above) ideals of $A$. We want to apply the chinese remainder theorem,
        but a priori we can't, because $M$ might be infinite. We claim however
        that in our situation, $M$ is finite. To show this, suppose
        that $(\fm_1, \fm_2, \dots)$ be an infinite sequence of elements in
        $I$. By the chinese remainder theorem, there is for any $N \in \N$ an
        isomorphism
        \begin{equation*}
            A/(\fm_1\cap \dots \cap \fm_N) \cong A/\fm_1 \times \dots \times A/\fm_N.
        \end{equation*}
        The left-hand side has dimension $\leq \dim_k(A)$, as it is a quotient
        of $A$. Meanwhile, the right-hand side has dimension $\geq N$, as 
        every quotient $A/\fm_i$ is a non-trivial $k$-vector space and thereby
        has dimension at least $1$. If we choose $N > \dim_k(A)$, we arrive 
        at a contradiction. Now $M = \{\fm_1, \dots, \fm_n\}$ is finite, and applying
        the chinese remainder theorem again yields the desired decomposition. 
        All factors are field extensions of $k$ of degree $\leq \dim_k(A)$, in
        particular finite.
\end{enumerate}


\contactend
\end{document}
